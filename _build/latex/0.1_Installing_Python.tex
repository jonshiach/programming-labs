%% Generated by Sphinx.
\def\sphinxdocclass{jupyterBook}
\documentclass[letterpaper,10pt,english]{jupyterBook}
\ifdefined\pdfpxdimen
   \let\sphinxpxdimen\pdfpxdimen\else\newdimen\sphinxpxdimen
\fi \sphinxpxdimen=.75bp\relax
\ifdefined\pdfimageresolution
    \pdfimageresolution= \numexpr \dimexpr1in\relax/\sphinxpxdimen\relax
\fi
%% let collapsible pdf bookmarks panel have high depth per default
\PassOptionsToPackage{bookmarksdepth=5}{hyperref}
%% turn off hyperref patch of \index as sphinx.xdy xindy module takes care of
%% suitable \hyperpage mark-up, working around hyperref-xindy incompatibility
\PassOptionsToPackage{hyperindex=false}{hyperref}
%% memoir class requires extra handling
\makeatletter\@ifclassloaded{memoir}
{\ifdefined\memhyperindexfalse\memhyperindexfalse\fi}{}\makeatother

\PassOptionsToPackage{warn}{textcomp}

\catcode`^^^^00a0\active\protected\def^^^^00a0{\leavevmode\nobreak\ }
\usepackage{cmap}
\usepackage{fontspec}
\defaultfontfeatures[\rmfamily,\sffamily,\ttfamily]{}
\usepackage{amsmath,amssymb,amstext}
\usepackage{polyglossia}
\setmainlanguage{english}



\setmainfont{FreeSerif}[
  Extension      = .otf,
  UprightFont    = *,
  ItalicFont     = *Italic,
  BoldFont       = *Bold,
  BoldItalicFont = *BoldItalic
]
\setsansfont{FreeSans}[
  Extension      = .otf,
  UprightFont    = *,
  ItalicFont     = *Oblique,
  BoldFont       = *Bold,
  BoldItalicFont = *BoldOblique,
]
\setmonofont{FreeMono}[
  Extension      = .otf,
  UprightFont    = *,
  ItalicFont     = *Oblique,
  BoldFont       = *Bold,
  BoldItalicFont = *BoldOblique,
]



\usepackage[Bjarne]{fncychap}
\usepackage[,numfigreset=1,mathnumfig]{sphinx}

\fvset{fontsize=\small}
\usepackage{geometry}


% Include hyperref last.
\usepackage{hyperref}
% Fix anchor placement for figures with captions.
\usepackage{hypcap}% it must be loaded after hyperref.
% Set up styles of URL: it should be placed after hyperref.
\urlstyle{same}


\usepackage{sphinxmessages}



        % Start of preamble defined in sphinx-jupyterbook-latex %
         \usepackage[Latin,Greek]{ucharclasses}
        \usepackage{unicode-math}
        % fixing title of the toc
        \addto\captionsenglish{\renewcommand{\contentsname}{Contents}}
        \hypersetup{
            pdfencoding=auto,
            psdextra
        }
        % End of preamble defined in sphinx-jupyterbook-latex %
        

\title{Installing Python}
\date{Sep 25, 2024}
\release{}
\author{Dr Jon Shiach}
\newcommand{\sphinxlogo}{\vbox{}}
\renewcommand{\releasename}{}
\makeindex
\begin{document}

\pagestyle{empty}
\sphinxmaketitle
\pagestyle{plain}
\sphinxtableofcontents
\pagestyle{normal}
\phantomsection\label{\detokenize{_pages/0.1_Installing_Python::doc}}


\sphinxAtStartPar
Python is installed on the PCs in the Dalton Building labs. To work on these materials at home you will need to download and install Python to your own computer. Fortunately Python is open source meaning that it is free to download and install.


\part{For Windows}
\label{\detokenize{_pages/0.1_Installing_Python:for-windows}}\begin{enumerate}
\sphinxsetlistlabels{\arabic}{enumi}{enumii}{}{.}%
\item {} 
\sphinxAtStartPar
Download the Installer:
\begin{itemize}
\item {} 
\sphinxAtStartPar
Go to the official Python website at www.python.org

\item {} 
\sphinxAtStartPar
Click on the \sphinxstylestrong{Downloads} tab

\item {} 
\sphinxAtStartPar
Click on the \sphinxstylestrong{Download Python X.X.X} button (where X.X.X is the latest version)

\end{itemize}

\item {} 
\sphinxAtStartPar
Run the Installer:
\begin{itemize}
\item {} 
\sphinxAtStartPar
Locate the downloaded file (usually in your Downloads folder) and double\sphinxhyphen{}click on it

\item {} 
\sphinxAtStartPar
In the installer window, check the box that says \sphinxstylestrong{Add Python X.X to PATH} at the bottom

\item {} 
\sphinxAtStartPar
Click on \sphinxstylestrong{Install Now}

\end{itemize}

\item {} 
\sphinxAtStartPar
Installation Process:
\begin{itemize}
\item {} 
\sphinxAtStartPar
The installer will begin installing Python, this process includes installing pip (Python’s package installer) and other necessary components

\item {} 
\sphinxAtStartPar
Once the installation is complete, click on \sphinxstylestrong{Close}

\end{itemize}

\item {} 
\sphinxAtStartPar
Verify the Installation:
\begin{itemize}
\item {} 
\sphinxAtStartPar
Open the Command Prompt (you can do this by typing \sphinxcode{\sphinxupquote{cmd}} in the Start menu and pressing Enter)

\item {} 
\sphinxAtStartPar
Type \sphinxcode{\sphinxupquote{python \sphinxhyphen{}\sphinxhyphen{}version}} and press Enter, you should see a message displaying the Python version you installed

\item {} 
\sphinxAtStartPar
Similarly, type \sphinxcode{\sphinxupquote{pip \sphinxhyphen{}\sphinxhyphen{}version}} to verify pip installation.

\end{itemize}

\end{enumerate}


\part{For macOS}
\label{\detokenize{_pages/0.1_Installing_Python:for-macos}}\begin{enumerate}
\sphinxsetlistlabels{\arabic}{enumi}{enumii}{}{.}%
\item {} 
\sphinxAtStartPar
Download the Installer:
\begin{itemize}
\item {} 
\sphinxAtStartPar
Go to the official Python website at www.python.org

\item {} 
\sphinxAtStartPar
Click on the \sphinxstylestrong{Downloads} tab

\item {} 
\sphinxAtStartPar
Click on the \sphinxstylestrong{Download Python X.X.X} button (where X.X.X is the latest version)

\end{itemize}

\item {} 
\sphinxAtStartPar
Run the Installer:
\begin{itemize}
\item {} 
\sphinxAtStartPar
Locate the downloaded .pkg file (usually in your Downloads folder) and double\sphinxhyphen{}click on it

\item {} 
\sphinxAtStartPar
Follow the instructions in the installer, clicking \sphinxstylestrong{Continue} and \sphinxstylestrong{Install} as needed

\end{itemize}

\item {} 
\sphinxAtStartPar
Installation Process:
\begin{itemize}
\item {} 
\sphinxAtStartPar
The installer will guide you through the installation process, including installing IDLE, pip, and setting up the PATH

\end{itemize}

\item {} 
\sphinxAtStartPar
Verify the Installation:
\begin{itemize}
\item {} 
\sphinxAtStartPar
Open the Terminal (you can do this by searching for Terminal in Spotlight)

\item {} 
\sphinxAtStartPar
Type \sphinxcode{\sphinxupquote{python3 \sphinxhyphen{}\sphinxhyphen{}version}} and press Enter, you should see a message displaying the Python version you installed

\item {} 
\sphinxAtStartPar
Similarly, type \sphinxcode{\sphinxupquote{pip3 \sphinxhyphen{}\sphinxhyphen{}version}} to verify pip installation

\end{itemize}

\end{enumerate}


\part{Installing Spyder}
\label{\detokenize{_pages/0.1_Installing_Python:installing-spyder}}
\sphinxAtStartPar
Spyder is an open\sphinxhyphen{}source integrated development environment (IDE) specifically designed for scientific computing, data analysis, and research in Python. It provides a comprehensive set of tools that facilitate efficient code development, debugging, and exploration of data through a user\sphinxhyphen{}friendly interface.
\begin{enumerate}
\sphinxsetlistlabels{\arabic}{enumi}{enumii}{}{.}%
\item {} 
\sphinxAtStartPar
Go to https://www.spyder\sphinxhyphen{}ide.org

\item {} 
\sphinxAtStartPar
Click on the appropriate download link for your computer (if you have a newer mac choose “Download for macOS(M1)”)

\item {} 
\sphinxAtStartPar
Open the downloaded installer program and follow the onscreen instructions.

\end{enumerate}







\renewcommand{\indexname}{Index}
\printindex
\end{document}