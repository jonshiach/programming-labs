%% Generated by Sphinx.
\def\sphinxdocclass{jupyterBook}
\documentclass[letterpaper,10pt,english]{jupyterBook}
\ifdefined\pdfpxdimen
   \let\sphinxpxdimen\pdfpxdimen\else\newdimen\sphinxpxdimen
\fi \sphinxpxdimen=.75bp\relax
\ifdefined\pdfimageresolution
    \pdfimageresolution= \numexpr \dimexpr1in\relax/\sphinxpxdimen\relax
\fi
%% let collapsible pdf bookmarks panel have high depth per default
\PassOptionsToPackage{bookmarksdepth=5}{hyperref}
%% turn off hyperref patch of \index as sphinx.xdy xindy module takes care of
%% suitable \hyperpage mark-up, working around hyperref-xindy incompatibility
\PassOptionsToPackage{hyperindex=false}{hyperref}
%% memoir class requires extra handling
\makeatletter\@ifclassloaded{memoir}
{\ifdefined\memhyperindexfalse\memhyperindexfalse\fi}{}\makeatother

\PassOptionsToPackage{warn}{textcomp}

\catcode`^^^^00a0\active\protected\def^^^^00a0{\leavevmode\nobreak\ }
\usepackage{cmap}
\usepackage{fontspec}
\defaultfontfeatures[\rmfamily,\sffamily,\ttfamily]{}
\usepackage{amsmath,amssymb,amstext}
\usepackage{polyglossia}
\setmainlanguage{english}



\setmainfont{FreeSerif}[
  Extension      = .otf,
  UprightFont    = *,
  ItalicFont     = *Italic,
  BoldFont       = *Bold,
  BoldItalicFont = *BoldItalic
]
\setsansfont{FreeSans}[
  Extension      = .otf,
  UprightFont    = *,
  ItalicFont     = *Oblique,
  BoldFont       = *Bold,
  BoldItalicFont = *BoldOblique,
]
\setmonofont{FreeMono}[
  Extension      = .otf,
  UprightFont    = *,
  ItalicFont     = *Oblique,
  BoldFont       = *Bold,
  BoldItalicFont = *BoldOblique,
]



\usepackage[Bjarne]{fncychap}
\usepackage[,numfigreset=1,mathnumfig]{sphinx}

\fvset{fontsize=\small}
\usepackage{geometry}


% Include hyperref last.
\usepackage{hyperref}
% Fix anchor placement for figures with captions.
\usepackage{hypcap}% it must be loaded after hyperref.
% Set up styles of URL: it should be placed after hyperref.
\urlstyle{same}


\usepackage{sphinxmessages}



        % Start of preamble defined in sphinx-jupyterbook-latex %
         \usepackage[Latin,Greek]{ucharclasses}
        \usepackage{unicode-math}
        % fixing title of the toc
        \addto\captionsenglish{\renewcommand{\contentsname}{Contents}}
        \hypersetup{
            pdfencoding=auto,
            psdextra
        }
        % End of preamble defined in sphinx-jupyterbook-latex %
        

\title{Python Data Types}
\date{Sep 25, 2024}
\release{}
\author{Dr Jon Shiach}
\newcommand{\sphinxlogo}{\vbox{}}
\renewcommand{\releasename}{}
\makeindex
\begin{document}

\pagestyle{empty}
\sphinxmaketitle
\pagestyle{plain}
\sphinxtableofcontents
\pagestyle{normal}
\phantomsection\label{\detokenize{_pages/1.4_Python_data_types::doc}}


\sphinxAtStartPar
Variables can store different types of data which are used to do different things.


\begin{savenotes}\sphinxattablestart
\centering
\sphinxcapstartof{table}
\sphinxthecaptionisattop
\sphinxcaption{Python Data Types}\label{\detokenize{_pages/1.4_Python_data_types:data-type-table}}
\sphinxaftertopcaption
\begin{tabulary}{\linewidth}[t]{|T|T|T|}
\hline
\sphinxstyletheadfamily 
\sphinxAtStartPar
Data type
&\sphinxstyletheadfamily 
\sphinxAtStartPar
Class
&\sphinxstyletheadfamily 
\sphinxAtStartPar
Description
\\
\hline
\sphinxAtStartPar
Integer
&
\sphinxAtStartPar
\sphinxcode{\sphinxupquote{int}}
&
\sphinxAtStartPar
Integer values, e.g., \sphinxcode{\sphinxupquote{\sphinxhyphen{}1}}, \sphinxcode{\sphinxupquote{0}}, \sphinxcode{\sphinxupquote{1}}, \sphinxcode{\sphinxupquote{99}}
\\
\hline
\sphinxAtStartPar
Float
&
\sphinxAtStartPar
\sphinxcode{\sphinxupquote{float}}
&
\sphinxAtStartPar
Floating point values (decimals), e.g., \sphinxcode{\sphinxupquote{1.5}}, \sphinxcode{\sphinxupquote{2.718}}, \sphinxcode{\sphinxupquote{3.1415927}}
\\
\hline
\sphinxAtStartPar
Complex
&
\sphinxAtStartPar
\sphinxcode{\sphinxupquote{complex}}
&
\sphinxAtStartPar
Complex numbers, e.g., \sphinxcode{\sphinxupquote{1 + 2j}} (note the use of \sphinxcode{\sphinxupquote{j}} for the imaginary number)
\\
\hline
\sphinxAtStartPar
Boolean
&
\sphinxAtStartPar
\sphinxcode{\sphinxupquote{bool}}
&
\sphinxAtStartPar
Boolean values, e.g., \sphinxcode{\sphinxupquote{True}} or \sphinxcode{\sphinxupquote{False}}
\\
\hline
\sphinxAtStartPar
String
&
\sphinxAtStartPar
\sphinxcode{\sphinxupquote{str}}
&
\sphinxAtStartPar
Character string, e.g., \sphinxcode{\sphinxupquote{"hello world"}}
\\
\hline
\sphinxAtStartPar
List
&
\sphinxAtStartPar
\sphinxcode{\sphinxupquote{list}}
&
\sphinxAtStartPar
A list of elements, e.g., \sphinxcode{\sphinxupquote{{[}1, 1.5, "hello", 1 + 2j{]}}}
\\
\hline
\end{tabulary}
\par
\sphinxattableend\end{savenotes}

\sphinxAtStartPar
Python automatically use the appropriate data type for the value being assigned to a variable. The data type can be determined using the \sphinxcode{\sphinxupquote{type()}} command.

\begin{sphinxVerbatim}[commandchars=\\\{\}]
type(variable)
\end{sphinxVerbatim}

\sphinxAtStartPar
Enter the following commands into the console.

\begin{sphinxVerbatim}[commandchars=\\\{\}]
\PYG{n}{In} \PYG{p}{[}\PYG{l+m+mi}{30}\PYG{p}{]}\PYG{p}{:} \PYG{n+nb}{type}\PYG{p}{(}\PYG{o}{\PYGZhy{}}\PYG{l+m+mi}{1}\PYG{p}{)}
\PYG{n}{Out}\PYG{p}{[}\PYG{l+m+mi}{30}\PYG{p}{]}\PYG{p}{:} \PYG{n+nb}{int}

\PYG{n}{In} \PYG{p}{[}\PYG{l+m+mi}{31}\PYG{p}{]}\PYG{p}{:} \PYG{n+nb}{type}\PYG{p}{(}\PYG{l+m+mf}{1.5}\PYG{p}{)}
\PYG{n}{Out}\PYG{p}{[}\PYG{l+m+mi}{31}\PYG{p}{]}\PYG{p}{:} \PYG{n+nb}{float}

\PYG{n}{In} \PYG{p}{[}\PYG{l+m+mi}{32}\PYG{p}{]}\PYG{p}{:} \PYG{n+nb}{type}\PYG{p}{(}\PYG{l+m+mi}{1} \PYG{o}{+} \PYG{l+m+mi}{2}\PYG{n}{j}\PYG{p}{)}
\PYG{n}{Out}\PYG{p}{[}\PYG{l+m+mi}{32}\PYG{p}{]}\PYG{p}{:} \PYG{n+nb}{complex}

\PYG{n}{In} \PYG{p}{[}\PYG{l+m+mi}{33}\PYG{p}{]}\PYG{p}{:} \PYG{n+nb}{type}\PYG{p}{(}\PYG{k+kc}{True}\PYG{p}{)}
\PYG{n}{Out}\PYG{p}{[}\PYG{l+m+mi}{33}\PYG{p}{]}\PYG{p}{:} \PYG{n+nb}{bool}

\PYG{n}{In} \PYG{p}{[}\PYG{l+m+mi}{34}\PYG{p}{]}\PYG{p}{:} \PYG{n+nb}{type}\PYG{p}{(}\PYG{l+s+s2}{\PYGZdq{}}\PYG{l+s+s2}{Hello}\PYG{l+s+s2}{\PYGZdq{}}\PYG{p}{)}
\PYG{n}{Out}\PYG{p}{[}\PYG{l+m+mi}{34}\PYG{p}{]}\PYG{p}{:} \PYG{n+nb}{str}

\PYG{n}{In} \PYG{p}{[}\PYG{l+m+mi}{35}\PYG{p}{]}\PYG{p}{:} \PYG{n+nb}{type}\PYG{p}{(}\PYG{p}{[}\PYG{l+m+mi}{1}\PYG{p}{,} \PYG{l+m+mf}{1.5}\PYG{p}{,} \PYG{l+s+s2}{\PYGZdq{}}\PYG{l+s+s2}{hello}\PYG{l+s+s2}{\PYGZdq{}}\PYG{p}{,} \PYG{l+m+mi}{1} \PYG{o}{+} \PYG{l+m+mi}{2}\PYG{n}{j}\PYG{p}{]}\PYG{p}{)}
\PYG{n}{Out}\PYG{p}{[}\PYG{l+m+mi}{35}\PYG{p}{]}\PYG{p}{:} \PYG{n+nb}{list}
\end{sphinxVerbatim}


\part{Casting data types}
\label{\detokenize{_pages/1.4_Python_data_types:casting-data-types}}
\sphinxAtStartPar
We can change the data type for a variable using \sphinxstylestrong{casting} with the following functions
\begin{itemize}
\item {} 
\sphinxAtStartPar
\sphinxcode{\sphinxupquote{int()}} \sphinxhyphen{} returns an integer number from an input of a floating point number or string

\item {} 
\sphinxAtStartPar
\sphinxcode{\sphinxupquote{float()}} \sphinxhyphen{} returns a floating point number from an input of an integer or string

\item {} 
\sphinxAtStartPar
\sphinxcode{\sphinxupquote{str()}} \sphinxhyphen{} returns a string from a range of data types.

\end{itemize}

\sphinxAtStartPar
Lets try casting between different data types. Enter the following commands into the console.

\begin{sphinxVerbatim}[commandchars=\\\{\}]
\PYG{n}{In} \PYG{p}{[}\PYG{l+m+mi}{36}\PYG{p}{]}\PYG{p}{:} \PYG{n+nb}{int}\PYG{p}{(}\PYG{l+m+mf}{1.23}\PYG{p}{)}
\PYG{n}{Out}\PYG{p}{[}\PYG{l+m+mi}{36}\PYG{p}{]}\PYG{p}{:} \PYG{l+m+mi}{1}

\PYG{n}{In} \PYG{p}{[}\PYG{l+m+mi}{37}\PYG{p}{]}\PYG{p}{:} \PYG{n+nb}{int}\PYG{p}{(}\PYG{l+s+s2}{\PYGZdq{}}\PYG{l+s+s2}{123}\PYG{l+s+s2}{\PYGZdq{}}\PYG{p}{)}
\PYG{n}{Out}\PYG{p}{[}\PYG{l+m+mi}{37}\PYG{p}{]}\PYG{p}{:} \PYG{l+m+mi}{123}

\PYG{n}{In} \PYG{p}{[}\PYG{l+m+mi}{38}\PYG{p}{]}\PYG{p}{:} \PYG{n+nb}{float}\PYG{p}{(}\PYG{l+m+mi}{123}\PYG{p}{)}
\PYG{n}{Out}\PYG{p}{[}\PYG{l+m+mi}{38}\PYG{p}{]}\PYG{p}{:} \PYG{l+m+mf}{123.0}

\PYG{n}{In} \PYG{p}{[}\PYG{l+m+mi}{39}\PYG{p}{]}\PYG{p}{:} \PYG{n+nb}{float}\PYG{p}{(}\PYG{l+s+s2}{\PYGZdq{}}\PYG{l+s+s2}{123}\PYG{l+s+s2}{\PYGZdq{}}\PYG{p}{)}
\PYG{n}{Out}\PYG{p}{[}\PYG{l+m+mi}{39}\PYG{p}{]}\PYG{p}{:} \PYG{l+m+mf}{123.0}

\PYG{n}{In} \PYG{p}{[}\PYG{l+m+mi}{40}\PYG{p}{]}\PYG{p}{:} \PYG{n+nb}{str}\PYG{p}{(}\PYG{l+m+mi}{123}\PYG{p}{)}
\PYG{n}{Out}\PYG{p}{[}\PYG{l+m+mi}{40}\PYG{p}{]}\PYG{p}{:} \PYG{l+s+s1}{\PYGZsq{}}\PYG{l+s+s1}{123}\PYG{l+s+s1}{\PYGZsq{}}

\PYG{n}{In} \PYG{p}{[}\PYG{l+m+mi}{41}\PYG{p}{]}\PYG{p}{:} \PYG{n+nb}{str}\PYG{p}{(}\PYG{l+m+mf}{1.23}\PYG{p}{)}
\PYG{n}{Out}\PYG{p}{[}\PYG{l+m+mi}{41}\PYG{p}{]}\PYG{p}{:} \PYG{l+s+s1}{\PYGZsq{}}\PYG{l+s+s1}{1.23}\PYG{l+s+s1}{\PYGZsq{}}

\PYG{n}{In} \PYG{p}{[}\PYG{l+m+mi}{42}\PYG{p}{]}\PYG{p}{:} \PYG{n+nb}{str}\PYG{p}{(}\PYG{l+m+mi}{1} \PYG{o}{+} \PYG{l+m+mi}{2}\PYG{n}{j}\PYG{p}{)}
\PYG{n}{Out}\PYG{p}{[}\PYG{l+m+mi}{42}\PYG{p}{]}\PYG{p}{:} \PYG{l+s+s1}{\PYGZsq{}}\PYG{l+s+s1}{(1+2j)}\PYG{l+s+s1}{\PYGZsq{}}
\end{sphinxVerbatim}







\renewcommand{\indexname}{Index}
\printindex
\end{document}