%% Generated by Sphinx.
\def\sphinxdocclass{jupyterBook}
\documentclass[letterpaper,10pt,english]{jupyterBook}
\ifdefined\pdfpxdimen
   \let\sphinxpxdimen\pdfpxdimen\else\newdimen\sphinxpxdimen
\fi \sphinxpxdimen=.75bp\relax
\ifdefined\pdfimageresolution
    \pdfimageresolution= \numexpr \dimexpr1in\relax/\sphinxpxdimen\relax
\fi
%% let collapsible pdf bookmarks panel have high depth per default
\PassOptionsToPackage{bookmarksdepth=5}{hyperref}
%% turn off hyperref patch of \index as sphinx.xdy xindy module takes care of
%% suitable \hyperpage mark-up, working around hyperref-xindy incompatibility
\PassOptionsToPackage{hyperindex=false}{hyperref}
%% memoir class requires extra handling
\makeatletter\@ifclassloaded{memoir}
{\ifdefined\memhyperindexfalse\memhyperindexfalse\fi}{}\makeatother

\PassOptionsToPackage{warn}{textcomp}

\catcode`^^^^00a0\active\protected\def^^^^00a0{\leavevmode\nobreak\ }
\usepackage{cmap}
\usepackage{fontspec}
\defaultfontfeatures[\rmfamily,\sffamily,\ttfamily]{}
\usepackage{amsmath,amssymb,amstext}
\usepackage{polyglossia}
\setmainlanguage{english}



\setmainfont{FreeSerif}[
  Extension      = .otf,
  UprightFont    = *,
  ItalicFont     = *Italic,
  BoldFont       = *Bold,
  BoldItalicFont = *BoldItalic
]
\setsansfont{FreeSans}[
  Extension      = .otf,
  UprightFont    = *,
  ItalicFont     = *Oblique,
  BoldFont       = *Bold,
  BoldItalicFont = *BoldOblique,
]
\setmonofont{FreeMono}[
  Extension      = .otf,
  UprightFont    = *,
  ItalicFont     = *Oblique,
  BoldFont       = *Bold,
  BoldItalicFont = *BoldOblique,
]



\usepackage[Bjarne]{fncychap}
\usepackage[,numfigreset=1,mathnumfig]{sphinx}

\fvset{fontsize=\small}
\usepackage{geometry}


% Include hyperref last.
\usepackage{hyperref}
% Fix anchor placement for figures with captions.
\usepackage{hypcap}% it must be loaded after hyperref.
% Set up styles of URL: it should be placed after hyperref.
\urlstyle{same}


\usepackage{sphinxmessages}



        % Start of preamble defined in sphinx-jupyterbook-latex %
         \usepackage[Latin,Greek]{ucharclasses}
        \usepackage{unicode-math}
        % fixing title of the toc
        \addto\captionsenglish{\renewcommand{\contentsname}{Contents}}
        \hypersetup{
            pdfencoding=auto,
            psdextra
        }
        % End of preamble defined in sphinx-jupyterbook-latex %
        

\title{Basic arithmetic operations}
\date{Sep 25, 2024}
\release{}
\author{Dr Jon Shiach}
\newcommand{\sphinxlogo}{\vbox{}}
\renewcommand{\releasename}{}
\makeindex
\begin{document}

\pagestyle{empty}
\sphinxmaketitle
\pagestyle{plain}
\sphinxtableofcontents
\pagestyle{normal}
\phantomsection\label{\detokenize{_pages/1.1_Basic_arithmetic_operations::doc}}


\sphinxAtStartPar
We will begin with using Python to perform basic arithmetic operations since these form the fundamentals of computer programming (it is useful to think of your computer as a very powerful calculator). The arithmetic operators used to perform the basic operations are shown in the table below.


\begin{savenotes}\sphinxattablestart
\centering
\sphinxcapstartof{table}
\sphinxthecaptionisattop
\sphinxcaption{Arithmetic operations in Python}\label{\detokenize{_pages/1.1_Basic_arithmetic_operations:arithmetic-operators-table}}
\sphinxaftertopcaption
\begin{tabulary}{\linewidth}[t]{|T|T|T|}
\hline
\sphinxstyletheadfamily 
\sphinxAtStartPar
Operation
&\sphinxstyletheadfamily 
\sphinxAtStartPar
Description
&\sphinxstyletheadfamily 
\sphinxAtStartPar
Python syntax
\\
\hline
\sphinxAtStartPar
\(x + y\)
&
\sphinxAtStartPar
addition
&
\sphinxAtStartPar
\sphinxcode{\sphinxupquote{x + y}}
\\
\hline
\sphinxAtStartPar
\(x - y\)
&
\sphinxAtStartPar
subtraction
&
\sphinxAtStartPar
\sphinxcode{\sphinxupquote{x \sphinxhyphen{} y}}
\\
\hline
\sphinxAtStartPar
\(x \times y\)
&
\sphinxAtStartPar
scalar multiplication
&
\sphinxAtStartPar
\sphinxcode{\sphinxupquote{x * y}}
\\
\hline
\sphinxAtStartPar
\(x \div y\)
&
\sphinxAtStartPar
scalar division
&
\sphinxAtStartPar
\sphinxcode{\sphinxupquote{x / y}}
\\
\hline
\sphinxAtStartPar
\(x^y\)
&
\sphinxAtStartPar
exponentiation (power)
&
\sphinxAtStartPar
\sphinxcode{\sphinxupquote{x ** y}}
\\
\hline
\sphinxAtStartPar
\(x \operatorname{mod} y\)
&
\sphinxAtStartPar
modulo (remainder)
&
\sphinxAtStartPar
\sphinxcode{\sphinxupquote{x \% y}}
\\
\hline
\sphinxAtStartPar
\(\lfloor x \div y \rfloor\)
&
\sphinxAtStartPar
integer division
&
\sphinxAtStartPar
\sphinxcode{\sphinxupquote{x // y}}
\\
\hline
\end{tabulary}
\par
\sphinxattableend\end{savenotes}

\sphinxAtStartPar
Lets use Python to perform the following calculations. In the \sphinxstylestrong{console} window in the bottom right\sphinxhyphen{}hand corner of the Spyder window enter the following commands, pressing the \sphinxstylestrong{Enter} key after each one.

\begin{sphinxVerbatim}[commandchars=\\\{\}]
\PYG{l+m+mi}{3} \PYG{o}{+} \PYG{l+m+mi}{4}
\PYG{l+m+mi}{4} \PYG{o}{\PYGZhy{}} \PYG{l+m+mi}{7}
\PYG{l+m+mi}{5} \PYG{o}{*} \PYG{l+m+mi}{3}
\PYG{l+m+mi}{2} \PYG{o}{/} \PYG{l+m+mi}{9}
\PYG{l+m+mi}{2} \PYG{o}{*}\PYG{o}{*} \PYG{l+m+mi}{5}
\PYG{l+m+mi}{23} \PYG{o}{\PYGZpc{}} \PYG{l+m+mi}{4}
\PYG{l+m+mi}{11} \PYG{o}{/}\PYG{o}{/} \PYG{l+m+mi}{3}
\end{sphinxVerbatim}

\sphinxAtStartPar
Your console should look like the following.

\begin{sphinxVerbatim}[commandchars=\\\{\}]
\PYG{n}{In} \PYG{p}{[}\PYG{l+m+mi}{1}\PYG{p}{]}\PYG{p}{:} \PYG{l+m+mi}{3} \PYG{o}{+} \PYG{l+m+mi}{4}
\PYG{n}{Out}\PYG{p}{[}\PYG{l+m+mi}{1}\PYG{p}{]}\PYG{p}{:} \PYG{l+m+mi}{7}

\PYG{n}{In} \PYG{p}{[}\PYG{l+m+mi}{2}\PYG{p}{]}\PYG{p}{:} \PYG{l+m+mi}{4} \PYG{o}{\PYGZhy{}} \PYG{l+m+mi}{7}
\PYG{n}{Out}\PYG{p}{[}\PYG{l+m+mi}{2}\PYG{p}{]}\PYG{p}{:} \PYG{o}{\PYGZhy{}}\PYG{l+m+mi}{3}

\PYG{n}{In} \PYG{p}{[}\PYG{l+m+mi}{3}\PYG{p}{]}\PYG{p}{:} \PYG{l+m+mi}{5} \PYG{o}{*} \PYG{l+m+mi}{3}
\PYG{n}{Out}\PYG{p}{[}\PYG{l+m+mi}{3}\PYG{p}{]}\PYG{p}{:} \PYG{l+m+mi}{15}

\PYG{n}{In} \PYG{p}{[}\PYG{l+m+mi}{4}\PYG{p}{]}\PYG{p}{:} \PYG{l+m+mi}{2} \PYG{o}{/} \PYG{l+m+mi}{9}
\PYG{n}{Out}\PYG{p}{[}\PYG{l+m+mi}{4}\PYG{p}{]}\PYG{p}{:} \PYG{l+m+mf}{0.2222222222222222}

\PYG{n}{In} \PYG{p}{[}\PYG{l+m+mi}{5}\PYG{p}{]}\PYG{p}{:} \PYG{l+m+mi}{2} \PYG{o}{*}\PYG{o}{*} \PYG{l+m+mi}{5}
\PYG{n}{Out}\PYG{p}{[}\PYG{l+m+mi}{5}\PYG{p}{]}\PYG{p}{:} \PYG{l+m+mi}{32}

\PYG{n}{In} \PYG{p}{[}\PYG{l+m+mi}{6}\PYG{p}{]}\PYG{p}{:} \PYG{l+m+mi}{23} \PYG{o}{\PYGZpc{}} \PYG{l+m+mi}{4}
\PYG{n}{Out}\PYG{p}{[}\PYG{l+m+mi}{6}\PYG{p}{]}\PYG{p}{:} \PYG{l+m+mi}{3}

\PYG{n}{In} \PYG{p}{[}\PYG{l+m+mi}{7}\PYG{p}{]}\PYG{p}{:} \PYG{l+m+mi}{11} \PYG{o}{/}\PYG{o}{/} \PYG{l+m+mi}{3}
\PYG{n}{Out}\PYG{p}{[}\PYG{l+m+mi}{7}\PYG{p}{]}\PYG{p}{:} \PYG{l+m+mi}{3}
\end{sphinxVerbatim}

\sphinxAtStartPar
Most of this should be fairly self explanatory. Don’t worry if the numbers in the square brackets do not match what you see in your console.

\begin{sphinxadmonition}{note}{Note:}
\sphinxAtStartPar
Note that \(2 \div 9 = 0.\dot{2}\) and Python has outputted the result using 16 decimal places. So this is an approximation of the actual value. Its important to realise that when we perform calculations on a computer, the results that are returned are approximations of the actual values.
\end{sphinxadmonition}


\part{Order of precedence of operations}
\label{\detokenize{_pages/1.1_Basic_arithmetic_operations:order-of-precedence-of-operations}}
\sphinxAtStartPar
Python follows the standard rules for order of operations, i.e., BIDMAS: Brackets > Indices (powers) > Division, Multiplication > Addition, Subtraction. Brackets should be used to override this where necessary.

\sphinxAtStartPar
For example, lets calculate the value of \(\dfrac{1}{2+3}\). Enter the following in to the console and press enter.

\begin{sphinxVerbatim}[commandchars=\\\{\}]
\PYG{n}{In} \PYG{p}{[}\PYG{l+m+mi}{8}\PYG{p}{]}\PYG{p}{:} \PYG{l+m+mi}{1} \PYG{o}{/} \PYG{p}{(}\PYG{l+m+mi}{2} \PYG{o}{+} \PYG{l+m+mi}{3}\PYG{p}{)}
\PYG{n}{Out}\PYG{p}{[}\PYG{l+m+mi}{8}\PYG{p}{]}\PYG{p}{:} \PYG{l+m+mf}{0.2}
\end{sphinxVerbatim}

\sphinxAtStartPar
Which is the correct result. What if we forget to include the brackets in the calculation. Enter the following into the console and press enter.

\begin{sphinxVerbatim}[commandchars=\\\{\}]
\PYG{n}{In} \PYG{p}{[}\PYG{l+m+mi}{9}\PYG{p}{]}\PYG{p}{:} \PYG{l+m+mi}{1} \PYG{o}{/} \PYG{l+m+mi}{2} \PYG{o}{+} \PYG{l+m+mi}{3}
\PYG{n}{Out}\PYG{p}{[}\PYG{l+m+mi}{9}\PYG{p}{]}\PYG{p}{:} \PYG{l+m+mf}{3.5}
\end{sphinxVerbatim}

\sphinxAtStartPar
Which is obviously incorrect. What Python has done here is calculate \(1 \div 2\) and then add 3 to the result. Care should be taken when dealing with arithmetic expressions that use multiple operators.


\bigskip\hrule\bigskip



\part{Exercise}
\label{\detokenize{_pages/1.1_Basic_arithmetic_operations:exercise}}\phantomsection \label{exercise:python-arithmetic-operations-ex}

\begin{sphinxadmonition}{note}{Exercise 1.2.1}



\sphinxAtStartPar
Use the console to calculate:
\begin{enumerate}
\sphinxsetlistlabels{\arabic}{enumi}{enumii}{}{.}%
\item {} 
\sphinxAtStartPar
\(2 - (3 + 6)\)

\item {} 
\sphinxAtStartPar
\(2(5 - 8(3 + 6))\)

\item {} 
\sphinxAtStartPar
\(2(2 - 2(3 - 6 + 5(4 - 7)))\)

\item {} 
\sphinxAtStartPar
\(\dfrac{2(5 - 4(3 + 8))}{3(4 - (3 - 5))}\)

\item {} 
\sphinxAtStartPar
\(\dfrac{2(4^5)}{81 - 5^2}\)

\item {} 
\sphinxAtStartPar
The remainder when 14151 is divided by 571

\item {} 
\sphinxAtStartPar
The number of times 1111 can be divided by 14

\end{enumerate}
\end{sphinxadmonition}







\renewcommand{\indexname}{Index}
\printindex
\end{document}