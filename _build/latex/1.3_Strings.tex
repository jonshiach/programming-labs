%% Generated by Sphinx.
\def\sphinxdocclass{jupyterBook}
\documentclass[letterpaper,10pt,english]{jupyterBook}
\ifdefined\pdfpxdimen
   \let\sphinxpxdimen\pdfpxdimen\else\newdimen\sphinxpxdimen
\fi \sphinxpxdimen=.75bp\relax
\ifdefined\pdfimageresolution
    \pdfimageresolution= \numexpr \dimexpr1in\relax/\sphinxpxdimen\relax
\fi
%% let collapsible pdf bookmarks panel have high depth per default
\PassOptionsToPackage{bookmarksdepth=5}{hyperref}
%% turn off hyperref patch of \index as sphinx.xdy xindy module takes care of
%% suitable \hyperpage mark-up, working around hyperref-xindy incompatibility
\PassOptionsToPackage{hyperindex=false}{hyperref}
%% memoir class requires extra handling
\makeatletter\@ifclassloaded{memoir}
{\ifdefined\memhyperindexfalse\memhyperindexfalse\fi}{}\makeatother

\PassOptionsToPackage{warn}{textcomp}

\catcode`^^^^00a0\active\protected\def^^^^00a0{\leavevmode\nobreak\ }
\usepackage{cmap}
\usepackage{fontspec}
\defaultfontfeatures[\rmfamily,\sffamily,\ttfamily]{}
\usepackage{amsmath,amssymb,amstext}
\usepackage{polyglossia}
\setmainlanguage{english}



\setmainfont{FreeSerif}[
  Extension      = .otf,
  UprightFont    = *,
  ItalicFont     = *Italic,
  BoldFont       = *Bold,
  BoldItalicFont = *BoldItalic
]
\setsansfont{FreeSans}[
  Extension      = .otf,
  UprightFont    = *,
  ItalicFont     = *Oblique,
  BoldFont       = *Bold,
  BoldItalicFont = *BoldOblique,
]
\setmonofont{FreeMono}[
  Extension      = .otf,
  UprightFont    = *,
  ItalicFont     = *Oblique,
  BoldFont       = *Bold,
  BoldItalicFont = *BoldOblique,
]



\usepackage[Bjarne]{fncychap}
\usepackage[,numfigreset=1,mathnumfig]{sphinx}

\fvset{fontsize=\small}
\usepackage{geometry}


% Include hyperref last.
\usepackage{hyperref}
% Fix anchor placement for figures with captions.
\usepackage{hypcap}% it must be loaded after hyperref.
% Set up styles of URL: it should be placed after hyperref.
\urlstyle{same}


\usepackage{sphinxmessages}



        % Start of preamble defined in sphinx-jupyterbook-latex %
         \usepackage[Latin,Greek]{ucharclasses}
        \usepackage{unicode-math}
        % fixing title of the toc
        \addto\captionsenglish{\renewcommand{\contentsname}{Contents}}
        \hypersetup{
            pdfencoding=auto,
            psdextra
        }
        % End of preamble defined in sphinx-jupyterbook-latex %
        

\title{Strings}
\date{Sep 25, 2024}
\release{}
\author{Dr Jon Shiach}
\newcommand{\sphinxlogo}{\vbox{}}
\renewcommand{\releasename}{}
\makeindex
\begin{document}

\pagestyle{empty}
\sphinxmaketitle
\pagestyle{plain}
\sphinxtableofcontents
\pagestyle{normal}
\phantomsection\label{\detokenize{_pages/1.3_Strings::doc}}


\sphinxAtStartPar
A \sphinxstylestrong{string} in programming is a sequence of characters and is used to handle text data. A string in Python is declared using single or double quotation marks, so \sphinxcode{\sphinxupquote{'hello world'}} is the same as \sphinxcode{\sphinxupquote{"hello world"}}.

\sphinxAtStartPar
Declaring a string variable is simply done by assigning a variable equal to a string.

\begin{sphinxVerbatim}[commandchars=\\\{\}]
variable = \PYGZdq{}string\PYGZdq{}
\end{sphinxVerbatim}

\sphinxAtStartPar
For example, enter the following into the console.

\begin{sphinxVerbatim}[commandchars=\\\{\}]
\PYG{n}{In} \PYG{p}{[}\PYG{l+m+mi}{30}\PYG{p}{]}\PYG{p}{:} \PYG{n}{string} \PYG{o}{=} \PYG{l+s+s2}{\PYGZdq{}}\PYG{l+s+s2}{hello world}\PYG{l+s+s2}{\PYGZdq{}}
\end{sphinxVerbatim}

\sphinxAtStartPar
To output a string we can use the \sphinxcode{\sphinxupquote{print()}} command (printing is explained in more detail \DUrole{xref,myst}{here}).

\begin{sphinxVerbatim}[commandchars=\\\{\}]
print(variable)
\end{sphinxVerbatim}

\sphinxAtStartPar
Lets print our \sphinxcode{\sphinxupquote{string}} variable, enter the following into the console

\begin{sphinxVerbatim}[commandchars=\\\{\}]
\PYG{n}{In} \PYG{p}{[}\PYG{l+m+mi}{31}\PYG{p}{]}\PYG{p}{:} \PYG{n+nb}{print}\PYG{p}{(}\PYG{n}{string}\PYG{p}{)}
\PYG{n}{Out}\PYG{p}{[}\PYG{l+m+mi}{31}\PYG{p}{]}\PYG{p}{:} \PYG{n}{hello} \PYG{n}{world}
\end{sphinxVerbatim}

\sphinxAtStartPar
To define a multiline string we can use triple single or double quotation marks. For example, enter the following into the console%
\begin{footnote}[1]\sphinxAtStartFootnote
This is a quote from the film \sphinxstyleemphasis{Monty Python and the Holy Grail} by the comedy group Monty Python. The creator of Python, Guido van Rossum, was looking for a short unique name and decided to name Python after Monty Python.
%
\end{footnote} (you will need to press the enter key after the word \sphinxcode{\sphinxupquote{velocity}} to continue onto the next line).

\begin{sphinxVerbatim}[commandchars=\\\{\}]
In [31]: multiline\PYGZus{}string = \PYGZdq{}\PYGZdq{}\PYGZdq{}What is the air\PYGZhy{}speed velocity 
    ...: of an unladen swallow?\PYGZdq{}\PYGZdq{}\PYGZdq{}

In [32]: print(multiline\PYGZus{}string)
What is the air\PYGZhy{}speed velocity 
of an unladen swallow?
\end{sphinxVerbatim}


\bigskip\hrule\bigskip



\part{Modifying strings}
\label{\detokenize{_pages/1.3_Strings:modifying-strings}}
\sphinxAtStartPar
Python has the following built\sphinxhyphen{}in functions that can be used to modify a string.


\begin{savenotes}\sphinxattablestart
\centering
\sphinxcapstartof{table}
\sphinxthecaptionisattop
\sphinxcaption{String modification functions}\label{\detokenize{_pages/1.3_Strings:id3}}
\sphinxaftertopcaption
\begin{tabulary}{\linewidth}[t]{|T|T|}
\hline
\sphinxstyletheadfamily 
\sphinxAtStartPar
Function
&\sphinxstyletheadfamily 
\sphinxAtStartPar
Description
\\
\hline
\sphinxAtStartPar
\sphinxcode{\sphinxupquote{string.upper()}}
&
\sphinxAtStartPar
Converts the characters of a string to uppercase
\\
\hline
\sphinxAtStartPar
\sphinxcode{\sphinxupquote{string.lower()}}
&
\sphinxAtStartPar
Converts the characters of a string to lowercase
\\
\hline
\sphinxAtStartPar
\sphinxcode{\sphinxupquote{string.strip()}}
&
\sphinxAtStartPar
Remove spaces before and after the characters in a string
\\
\hline
\sphinxAtStartPar
\sphinxcode{\sphinxupquote{string.replace(<old string>, <replacement string>)}}
&
\sphinxAtStartPar
Replaces a string with another string
\\
\hline
\end{tabulary}
\par
\sphinxattableend\end{savenotes}

\sphinxAtStartPar
To demonstrate these enter the following code into the console.

\begin{sphinxVerbatim}[commandchars=\\\{\}]
\PYG{n}{In} \PYG{p}{[}\PYG{l+m+mi}{33}\PYG{p}{]}\PYG{p}{:} \PYG{n}{string} \PYG{o}{=} \PYG{l+s+s2}{\PYGZdq{}}\PYG{l+s+s2}{   Hello World   }\PYG{l+s+s2}{\PYGZdq{}}

\PYG{n}{In} \PYG{p}{[}\PYG{l+m+mi}{33}\PYG{p}{]}\PYG{p}{:} \PYG{n+nb}{print}\PYG{p}{(}\PYG{n}{string}\PYG{o}{.}\PYG{n}{upper}\PYG{p}{(}\PYG{p}{)}\PYG{p}{)}
   \PYG{n}{HELLO} \PYG{n}{WORLD}   

\PYG{n}{In} \PYG{p}{[}\PYG{l+m+mi}{32}\PYG{p}{]}\PYG{p}{:} \PYG{n+nb}{print}\PYG{p}{(}\PYG{n}{string}\PYG{o}{.}\PYG{n}{lower}\PYG{p}{(}\PYG{p}{)}\PYG{p}{)}
   \PYG{n}{hello} \PYG{n}{world}   

\PYG{n}{In} \PYG{p}{[}\PYG{l+m+mi}{33}\PYG{p}{]}\PYG{p}{:} \PYG{n+nb}{print}\PYG{p}{(}\PYG{n}{string}\PYG{o}{.}\PYG{n}{strip}\PYG{p}{(}\PYG{p}{)}\PYG{p}{)}
\PYG{n}{Hello} \PYG{n}{World}

\PYG{n}{In} \PYG{p}{[}\PYG{l+m+mi}{34}\PYG{p}{]}\PYG{p}{:} \PYG{n+nb}{print}\PYG{p}{(}\PYG{n}{string}\PYG{o}{.}\PYG{n}{replace}\PYG{p}{(}\PYG{l+s+s2}{\PYGZdq{}}\PYG{l+s+s2}{l}\PYG{l+s+s2}{\PYGZdq{}}\PYG{p}{,} \PYG{l+s+s2}{\PYGZdq{}}\PYG{l+s+s2}{x}\PYG{l+s+s2}{\PYGZdq{}}\PYG{p}{)}\PYG{p}{)}
   \PYG{n}{Hexxo} \PYG{n}{Worxd}   
\end{sphinxVerbatim}


\bigskip\hrule\bigskip



\part{Concatenating strings}
\label{\detokenize{_pages/1.3_Strings:concatenating-strings}}
\sphinxAtStartPar
To concatenate (merge) two or more strings we use the \sphinxcode{\sphinxupquote{+}} operator.

\begin{sphinxVerbatim}[commandchars=\\\{\}]
merged\PYGZus{}string = string1 + string2 
\end{sphinxVerbatim}

\sphinxAtStartPar
To demonstrate this enter the following into the console.

\begin{sphinxVerbatim}[commandchars=\\\{\}]
\PYG{n}{In} \PYG{p}{[}\PYG{l+m+mi}{35}\PYG{p}{]}\PYG{p}{:} \PYG{n}{string1} \PYG{o}{=} \PYG{l+s+s2}{\PYGZdq{}}\PYG{l+s+s2}{hello}\PYG{l+s+s2}{\PYGZdq{}}

\PYG{n}{In} \PYG{p}{[}\PYG{l+m+mi}{36}\PYG{p}{]}\PYG{p}{:} \PYG{n}{string2} \PYG{o}{=} \PYG{l+s+s2}{\PYGZdq{}}\PYG{l+s+s2}{world}\PYG{l+s+s2}{\PYGZdq{}}

\PYG{n}{In} \PYG{p}{[}\PYG{l+m+mi}{37}\PYG{p}{]}\PYG{p}{:} \PYG{n}{merged\PYGZus{}string} \PYG{o}{=} \PYG{n}{string1} \PYG{o}{+} \PYG{l+s+s2}{\PYGZdq{}}\PYG{l+s+s2}{ }\PYG{l+s+s2}{\PYGZdq{}} \PYG{o}{+} \PYG{n}{string2}

\PYG{n}{In} \PYG{p}{[}\PYG{l+m+mi}{38}\PYG{p}{]}\PYG{p}{:} \PYG{n+nb}{print}\PYG{p}{(}\PYG{n}{merged\PYGZus{}string}\PYG{p}{)}
\PYG{n}{hello} \PYG{n}{world}
\end{sphinxVerbatim}

\sphinxAtStartPar
Note that we needed to include a space \sphinxcode{\sphinxupquote{" "}} when concatenating the two words, if we didn’t do this the concatenated string would be \sphinxcode{\sphinxupquote{helloworld}}.


\bigskip\hrule\bigskip



\part{Indexing characters in a string}
\label{\detokenize{_pages/1.3_Strings:indexing-characters-in-a-string}}
\sphinxAtStartPar
The characters in a string can be indexed using the character position starting at 0 for the first characeter.

\begin{sphinxVerbatim}[commandchars=\\\{\}]
string[ index ]
\end{sphinxVerbatim}

\sphinxAtStartPar
To demonstrate this enter the following into the console.

\begin{sphinxVerbatim}[commandchars=\\\{\}]
\PYG{n}{In} \PYG{p}{[}\PYG{l+m+mi}{39}\PYG{p}{]}\PYG{p}{:} \PYG{n}{string} \PYG{o}{=} \PYG{l+s+s2}{\PYGZdq{}}\PYG{l+s+s2}{What have the Romans ever done for us?}\PYG{l+s+s2}{\PYGZdq{}}

\PYG{n}{In} \PYG{p}{[}\PYG{l+m+mi}{40}\PYG{p}{]}\PYG{p}{:} \PYG{n+nb}{print}\PYG{p}{(}\PYG{n}{string}\PYG{p}{[}\PYG{l+m+mi}{0}\PYG{p}{]}\PYG{p}{)}
\PYG{n}{W}

\PYG{n}{In} \PYG{p}{[}\PYG{l+m+mi}{41}\PYG{p}{]}\PYG{p}{:} \PYG{n+nb}{print}\PYG{p}{(}\PYG{n}{string}\PYG{p}{[}\PYG{l+m+mi}{10}\PYG{p}{]}\PYG{p}{)}
\PYG{n}{t}
\end{sphinxVerbatim}

\sphinxAtStartPar
Here we have printed the 1st and 11th character in \sphinxcode{\sphinxupquote{string}}.

\sphinxAtStartPar
To index a range of characters in a string we use a colon to separate the first and last characters in the range.

\begin{sphinxVerbatim}[commandchars=\\\{\}]
string[ first\PYGZus{}character\PYGZus{}index : last\PYGZus{}character\PYGZus{}index + 1 ]
\end{sphinxVerbatim}

\sphinxAtStartPar
To demonstrate this enter the following into the console.

\begin{sphinxVerbatim}[commandchars=\\\{\}]
\PYG{n}{In} \PYG{p}{[}\PYG{l+m+mi}{42}\PYG{p}{]}\PYG{p}{:} \PYG{n+nb}{print}\PYG{p}{(}\PYG{n}{string}\PYG{p}{[}\PYG{l+m+mi}{14}\PYG{p}{:}\PYG{l+m+mi}{25}\PYG{p}{]}\PYG{p}{)}
\PYG{n}{Romans} \PYG{n}{ever}
\end{sphinxVerbatim}

\sphinxAtStartPar
Here we have printed the string which consists of the 15th to the 25th character in \sphinxcode{\sphinxupquote{string}}.

\begin{sphinxadmonition}{note}{Note:}
\sphinxAtStartPar
A Python string is actually an array of characters so we can use array slicing commands which are covered \DUrole{xref,myst}{later} to index strings.
\end{sphinxadmonition}


\bigskip\hrule\bigskip



\part{Length of a string}
\label{\detokenize{_pages/1.3_Strings:length-of-a-string}}
\sphinxAtStartPar
The length of a string is the number of characters in the string and can be determined using the \sphinxcode{\sphinxupquote{len()}} function.

\begin{sphinxVerbatim}[commandchars=\\\{\}]
len(string)
\end{sphinxVerbatim}

\sphinxAtStartPar
To demonstrate this enter the following into the console.

\begin{sphinxVerbatim}[commandchars=\\\{\}]
\PYG{n}{In} \PYG{p}{[}\PYG{l+m+mi}{43}\PYG{p}{]}\PYG{p}{:} \PYG{n+nb}{print}\PYG{p}{(}\PYG{n+nb}{len}\PYG{p}{(}\PYG{n}{string}\PYG{p}{)}\PYG{p}{)}
\PYG{l+m+mi}{38}
\end{sphinxVerbatim}

\sphinxAtStartPar
So our string is 38 characters long.


\bigskip\hrule\bigskip



\part{Exercise}
\label{\detokenize{_pages/1.3_Strings:exercise}}\phantomsection \label{exercise:python-strings-ex}

\begin{sphinxadmonition}{note}{Exercise 1.4.1}



\sphinxAtStartPar
Define two string variables for the following:
\begin{itemize}
\item {} 
\sphinxAtStartPar
String 1: “Your mother was a hamster”

\item {} 
\sphinxAtStartPar
String 2: “and your father smelt of elderberries!”

\end{itemize}

\sphinxAtStartPar
Use your strings to answer the following:
\begin{enumerate}
\sphinxsetlistlabels{\arabic}{enumi}{enumii}{}{.}%
\item {} 
\sphinxAtStartPar
Print string 1 using all lowercase characters

\item {} 
\sphinxAtStartPar
Print string 2 using all uppercase characters

\item {} 
\sphinxAtStartPar
Print string 2 with the word “elderberries” replaced with “roses”

\item {} 
\sphinxAtStartPar
Create another string variable by concatenating string 1 and string 2 and print it

\item {} 
\sphinxAtStartPar
Print the length of your concatenated string

\item {} 
\sphinxAtStartPar
Print the last 30 characters of the concatenated string

\end{enumerate}
\end{sphinxadmonition}


\bigskip\hrule\bigskip








\renewcommand{\indexname}{Index}
\printindex
\end{document}