%% Generated by Sphinx.
\def\sphinxdocclass{jupyterBook}
\documentclass[letterpaper,10pt,english]{jupyterBook}
\ifdefined\pdfpxdimen
   \let\sphinxpxdimen\pdfpxdimen\else\newdimen\sphinxpxdimen
\fi \sphinxpxdimen=.75bp\relax
\ifdefined\pdfimageresolution
    \pdfimageresolution= \numexpr \dimexpr1in\relax/\sphinxpxdimen\relax
\fi
%% let collapsible pdf bookmarks panel have high depth per default
\PassOptionsToPackage{bookmarksdepth=5}{hyperref}
%% turn off hyperref patch of \index as sphinx.xdy xindy module takes care of
%% suitable \hyperpage mark-up, working around hyperref-xindy incompatibility
\PassOptionsToPackage{hyperindex=false}{hyperref}
%% memoir class requires extra handling
\makeatletter\@ifclassloaded{memoir}
{\ifdefined\memhyperindexfalse\memhyperindexfalse\fi}{}\makeatother

\PassOptionsToPackage{warn}{textcomp}

\catcode`^^^^00a0\active\protected\def^^^^00a0{\leavevmode\nobreak\ }
\usepackage{cmap}
\usepackage{fontspec}
\defaultfontfeatures[\rmfamily,\sffamily,\ttfamily]{}
\usepackage{amsmath,amssymb,amstext}
\usepackage{polyglossia}
\setmainlanguage{english}



\setmainfont{FreeSerif}[
  Extension      = .otf,
  UprightFont    = *,
  ItalicFont     = *Italic,
  BoldFont       = *Bold,
  BoldItalicFont = *BoldItalic
]
\setsansfont{FreeSans}[
  Extension      = .otf,
  UprightFont    = *,
  ItalicFont     = *Oblique,
  BoldFont       = *Bold,
  BoldItalicFont = *BoldOblique,
]
\setmonofont{FreeMono}[
  Extension      = .otf,
  UprightFont    = *,
  ItalicFont     = *Oblique,
  BoldFont       = *Bold,
  BoldItalicFont = *BoldOblique,
]



\usepackage[Bjarne]{fncychap}
\usepackage[,numfigreset=1,mathnumfig]{sphinx}

\fvset{fontsize=\small}
\usepackage{geometry}


% Include hyperref last.
\usepackage{hyperref}
% Fix anchor placement for figures with captions.
\usepackage{hypcap}% it must be loaded after hyperref.
% Set up styles of URL: it should be placed after hyperref.
\urlstyle{same}


\usepackage{sphinxmessages}



        % Start of preamble defined in sphinx-jupyterbook-latex %
         \usepackage[Latin,Greek]{ucharclasses}
        \usepackage{unicode-math}
        % fixing title of the toc
        \addto\captionsenglish{\renewcommand{\contentsname}{Contents}}
        \hypersetup{
            pdfencoding=auto,
            psdextra
        }
        % End of preamble defined in sphinx-jupyterbook-latex %
        

\title{Formatting code}
\date{Sep 25, 2024}
\release{}
\author{Dr Jon Shiach}
\newcommand{\sphinxlogo}{\vbox{}}
\renewcommand{\releasename}{}
\makeindex
\begin{document}

\pagestyle{empty}
\sphinxmaketitle
\pagestyle{plain}
\sphinxtableofcontents
\pagestyle{normal}
\phantomsection\label{\detokenize{_pages/1.7_Formatting_code::doc}}


\sphinxAtStartPar
It is good programming practice to format your code using spaces, blank lines and commands so that it can be easily read. Guidelines for formatting Python code can be found in the PEP style guide for Python code and this section covers some of the basics.


\part{Comments}
\label{\detokenize{_pages/1.7_Formatting_code:comments}}
\sphinxAtStartPar
A comment in a program is text that is ignored by Python when the code is executed and are useful to helping people understand the program. Comments in Python can be entered in two ways. The first is to use the \sphinxcode{\sphinxupquote{\#}} symbol

\begin{sphinxVerbatim}[commandchars=\\\{\}]
\PYG{c+c1}{\PYGZsh{} this is a comment}
\end{sphinxVerbatim}

\sphinxAtStartPar
Here any text on the same line to the right of \sphinxcode{\sphinxupquote{\#}} is ignored. These are useful for short comments. For longer comments that span multiple lines we can use \sphinxcode{\sphinxupquote{"""}} to start and end a comment.

\begin{sphinxVerbatim}[commandchars=\\\{\}]
\PYG{l+s+sd}{\PYGZdq{}\PYGZdq{}\PYGZdq{}this is a comment}
\PYG{l+s+sd}{that spans multiple}
\PYG{l+s+sd}{lines\PYGZdq{}\PYGZdq{}\PYGZdq{}}
\end{sphinxVerbatim}

\sphinxAtStartPar
Comments are useful for telling someone (and yourself) what a program and individual lines of a program do. Add some comments to your program similar what is shown below.

\begin{sphinxVerbatim}[commandchars=\\\{\}]
\PYG{c+c1}{\PYGZsh{} 1. Python Basics}

\PYG{c+c1}{\PYGZsh{} Print the header}
\PYG{n+nb}{print}\PYG{p}{(}\PYG{l+s+s2}{\PYGZdq{}}\PYG{l+s+s2}{Degrees to radians conversion}\PYG{l+s+se}{\PYGZbs{}n}\PYG{l+s+s2}{\PYGZhy{}\PYGZhy{}\PYGZhy{}\PYGZhy{}\PYGZhy{}\PYGZhy{}\PYGZhy{}\PYGZhy{}\PYGZhy{}\PYGZhy{}\PYGZhy{}\PYGZhy{}\PYGZhy{}\PYGZhy{}\PYGZhy{}\PYGZhy{}\PYGZhy{}\PYGZhy{}\PYGZhy{}\PYGZhy{}\PYGZhy{}\PYGZhy{}\PYGZhy{}\PYGZhy{}\PYGZhy{}\PYGZhy{}\PYGZhy{}\PYGZhy{}\PYGZhy{}}\PYG{l+s+s2}{\PYGZdq{}}\PYG{p}{)}

\PYG{c+c1}{\PYGZsh{} Define the variables}
\PYG{n}{pi} \PYG{o}{=} \PYG{l+m+mf}{3.1415927}\PYG{p}{;}
\PYG{n}{angle\PYGZus{}in\PYGZus{}degrees} \PYG{o}{=} \PYG{l+m+mi}{45}\PYG{p}{;}

\PYG{c+c1}{\PYGZsh{} Calculate the angle in radians}
\PYG{n}{angle\PYGZus{}in\PYGZus{}radians} \PYG{o}{=} \PYG{n}{angle\PYGZus{}in\PYGZus{}degrees} \PYG{o}{*} \PYG{n}{pi} \PYG{o}{/} \PYG{l+m+mi}{180}\PYG{p}{;}

\PYG{c+c1}{\PYGZsh{} Print the angle in degrees and radians}
\PYG{n+nb}{print}\PYG{p}{(}\PYG{l+s+sa}{f}\PYG{l+s+s2}{\PYGZdq{}}\PYG{l+s+s2}{angle in degrees = }\PYG{l+s+si}{\PYGZob{}}\PYG{n}{angle\PYGZus{}in\PYGZus{}degrees}\PYG{l+s+si}{:}\PYG{l+s+s2}{6.3f}\PYG{l+s+si}{\PYGZcb{}}\PYG{l+s+s2}{\PYGZdq{}}\PYG{p}{)}
\PYG{n+nb}{print}\PYG{p}{(}\PYG{l+s+sa}{f}\PYG{l+s+s2}{\PYGZdq{}}\PYG{l+s+s2}{angle in radians = }\PYG{l+s+si}{\PYGZob{}}\PYG{n}{angle\PYGZus{}in\PYGZus{}radians}\PYG{l+s+si}{:}\PYG{l+s+s2}{6.3f}\PYG{l+s+si}{\PYGZcb{}}\PYG{l+s+s2}{\PYGZdq{}}\PYG{p}{)}
\end{sphinxVerbatim}


\part{Spaces}
\label{\detokenize{_pages/1.7_Formatting_code:spaces}}
\sphinxAtStartPar
In a Python program spaces are ignored, however it is common practice to use spaces either side of the arithmetic operators so that it is more readable. It is also advisable to separate blocks of code with a blank line.

\sphinxAtStartPar
For example the code

\begin{sphinxVerbatim}[commandchars=\\\{\}]
\PYG{l+m+mi}{1}\PYG{o}{*}\PYG{l+m+mi}{2}\PYG{o}{+}\PYG{l+m+mi}{3}\PYG{o}{*}\PYG{l+m+mi}{4}\PYG{o}{+}\PYG{l+m+mi}{5}\PYG{o}{/}\PYG{l+m+mi}{6}
\end{sphinxVerbatim}

\sphinxAtStartPar
is equivalent to the code

\begin{sphinxVerbatim}[commandchars=\\\{\}]
\PYG{l+m+mi}{1} \PYG{o}{*} \PYG{l+m+mi}{2} \PYG{o}{+} \PYG{l+m+mi}{3} \PYG{o}{*} \PYG{l+m+mi}{4} \PYG{o}{+} \PYG{l+m+mi}{5} \PYG{o}{/} \PYG{l+m+mi}{6}
\end{sphinxVerbatim}


\bigskip\hrule\bigskip



\part{Exercise}
\label{\detokenize{_pages/1.7_Formatting_code:exercise}}\phantomsection \label{exercise:python-formatting-code-ex}

\begin{sphinxadmonition}{note}{Exercise 1.8.1}



\sphinxAtStartPar
The repayments for a loan can be calculated using the formula
\begin{equation*}
\begin{split} P = \frac{rV}{1 - (1 + r)^{-n}}, \end{split}
\end{equation*}
\sphinxAtStartPar
where
\begin{itemize}
\item {} 
\sphinxAtStartPar
\(P\) is the monthly repayment

\item {} 
\sphinxAtStartPar
\(V\) is the value of the loan

\item {} 
\sphinxAtStartPar
\(r\) is the interest rate per period

\item {} 
\sphinxAtStartPar
\(n\) is the number of periods

\end{itemize}

\sphinxAtStartPar
The program below calculates the monthly repayments for a mortgage of value £200,000 taken out over 20 years with a fixed annual interest rate of 4\%.

\begin{sphinxVerbatim}[commandchars=\\\{\}]
\PYG{c+c1}{\PYGZsh{} Exercise 1.5}
\PYG{n}{a}\PYG{o}{=}\PYG{l+m+mi}{200000}
\PYG{n}{b}\PYG{o}{=}\PYG{l+m+mi}{20}
\PYG{n}{c}\PYG{o}{=}\PYG{l+m+mi}{4}
\PYG{n}{d}\PYG{o}{=}\PYG{l+m+mi}{12}\PYG{o}{*}\PYG{l+m+mi}{20}
\PYG{n}{e}\PYG{o}{=}\PYG{n}{c}\PYG{o}{/}\PYG{l+m+mi}{100}\PYG{o}{/}\PYG{l+m+mi}{12}
\PYG{n}{f}\PYG{o}{=}\PYG{n}{e}\PYG{o}{*}\PYG{n}{a}\PYG{o}{/}\PYG{p}{(}\PYG{l+m+mi}{1}\PYG{o}{\PYGZhy{}}\PYG{p}{(}\PYG{l+m+mi}{1}\PYG{o}{+}\PYG{n}{e}\PYG{o}{/}\PYG{l+m+mi}{100}\PYG{p}{)}\PYG{o}{*}\PYG{o}{*}\PYG{p}{(}\PYG{o}{\PYGZhy{}}\PYG{n}{d}\PYG{p}{)}\PYG{p}{)}
\PYG{n+nb}{print}\PYG{p}{(}\PYG{n}{a}\PYG{p}{)}
\PYG{n+nb}{print}\PYG{p}{(}\PYG{n}{b}\PYG{p}{)}
\PYG{n+nb}{print}\PYG{p}{(}\PYG{n}{c}\PYG{p}{)}
\PYG{n+nb}{print}\PYG{p}{(}\PYG{n}{f}\PYG{p}{)}
\end{sphinxVerbatim}

\sphinxAtStartPar
Copy and paste this program into your Python file and edit it to do the following:
\begin{enumerate}
\sphinxsetlistlabels{\arabic}{enumi}{enumii}{}{.}%
\item {} 
\sphinxAtStartPar
Replace variables \sphinxcode{\sphinxupquote{a}} to \sphinxcode{\sphinxupquote{f}} with suitable descriptive variable names

\item {} 
\sphinxAtStartPar
Use spaces to improve the code formatting

\item {} 
\sphinxAtStartPar
Add comments to briefly explain what the program is doing

\item {} 
\sphinxAtStartPar
Edit the print commands so that the program outputs the loan details and the monthly repayment amount in the following format

\end{enumerate}

\begin{sphinxVerbatim}[commandchars=\\\{\}]
Loan repayment calculator
\PYGZhy{}\PYGZhy{}\PYGZhy{}\PYGZhy{}\PYGZhy{}\PYGZhy{}\PYGZhy{}\PYGZhy{}\PYGZhy{}\PYGZhy{}\PYGZhy{}\PYGZhy{}\PYGZhy{}\PYGZhy{}\PYGZhy{}\PYGZhy{}\PYGZhy{}\PYGZhy{}\PYGZhy{}\PYGZhy{}\PYGZhy{}\PYGZhy{}\PYGZhy{}\PYGZhy{}\PYGZhy{}
Loan amount          : £xxxxxx.xx
Loan duration        : xx years
Annual interest rate : xx\PYGZpc{}

Monthly repayment    : £xxxx.xx
\end{sphinxVerbatim}
\end{sphinxadmonition}







\renewcommand{\indexname}{Index}
\printindex
\end{document}