%% Generated by Sphinx.
\def\sphinxdocclass{jupyterBook}
\documentclass[letterpaper,10pt,english]{jupyterBook}
\ifdefined\pdfpxdimen
   \let\sphinxpxdimen\pdfpxdimen\else\newdimen\sphinxpxdimen
\fi \sphinxpxdimen=.75bp\relax
\ifdefined\pdfimageresolution
    \pdfimageresolution= \numexpr \dimexpr1in\relax/\sphinxpxdimen\relax
\fi
%% let collapsible pdf bookmarks panel have high depth per default
\PassOptionsToPackage{bookmarksdepth=5}{hyperref}
%% turn off hyperref patch of \index as sphinx.xdy xindy module takes care of
%% suitable \hyperpage mark-up, working around hyperref-xindy incompatibility
\PassOptionsToPackage{hyperindex=false}{hyperref}
%% memoir class requires extra handling
\makeatletter\@ifclassloaded{memoir}
{\ifdefined\memhyperindexfalse\memhyperindexfalse\fi}{}\makeatother

\PassOptionsToPackage{warn}{textcomp}

\catcode`^^^^00a0\active\protected\def^^^^00a0{\leavevmode\nobreak\ }
\usepackage{cmap}
\usepackage{fontspec}
\defaultfontfeatures[\rmfamily,\sffamily,\ttfamily]{}
\usepackage{amsmath,amssymb,amstext}
\usepackage{polyglossia}
\setmainlanguage{english}



\setmainfont{FreeSerif}[
  Extension      = .otf,
  UprightFont    = *,
  ItalicFont     = *Italic,
  BoldFont       = *Bold,
  BoldItalicFont = *BoldItalic
]
\setsansfont{FreeSans}[
  Extension      = .otf,
  UprightFont    = *,
  ItalicFont     = *Oblique,
  BoldFont       = *Bold,
  BoldItalicFont = *BoldOblique,
]
\setmonofont{FreeMono}[
  Extension      = .otf,
  UprightFont    = *,
  ItalicFont     = *Oblique,
  BoldFont       = *Bold,
  BoldItalicFont = *BoldOblique,
]



\usepackage[Bjarne]{fncychap}
\usepackage[,numfigreset=1,mathnumfig]{sphinx}

\fvset{fontsize=\small}
\usepackage{geometry}


% Include hyperref last.
\usepackage{hyperref}
% Fix anchor placement for figures with captions.
\usepackage{hypcap}% it must be loaded after hyperref.
% Set up styles of URL: it should be placed after hyperref.
\urlstyle{same}


\usepackage{sphinxmessages}



        % Start of preamble defined in sphinx-jupyterbook-latex %
         \usepackage[Latin,Greek]{ucharclasses}
        \usepackage{unicode-math}
        % fixing title of the toc
        \addto\captionsenglish{\renewcommand{\contentsname}{Contents}}
        \hypersetup{
            pdfencoding=auto,
            psdextra
        }
        % End of preamble defined in sphinx-jupyterbook-latex %
        

\title{Python programs}
\date{Sep 25, 2024}
\release{}
\author{Dr Jon Shiach}
\newcommand{\sphinxlogo}{\vbox{}}
\renewcommand{\releasename}{}
\makeindex
\begin{document}

\pagestyle{empty}
\sphinxmaketitle
\pagestyle{plain}
\sphinxtableofcontents
\pagestyle{normal}
\phantomsection\label{\detokenize{_pages/1.5_Python_programs::doc}}


\sphinxAtStartPar
We have used the console for a while now and you may start to notice that it has some shortcomings. For example, if you want to change the value of a variable, any other commands that use that variable will need to be entered again. This is where programs are useful. A program is a file or a collection of files that contain Python commands that can be run.

\sphinxAtStartPar
We are going to use Spyder to write our programs. In Spyder create a new file:
\begin{enumerate}
\sphinxsetlistlabels{\arabic}{enumi}{enumii}{}{.}%
\item {} 
\sphinxAtStartPar
Click on \sphinxstylestrong{File > New file…}. This will create a new file with the file name  \sphinxstylestrong{\sphinxhref{http://untitled0.py}{untitled0.py}} (Python programs have the file extension .py).

\item {} 
\sphinxAtStartPar
Click on \sphinxstylestrong{File > Save as…}, navigate to the folder where you want to save the program (e.g., \sphinxstylestrong{Documents/Programming\_skills/}) and give it the filename \sphinxstylestrong{1\_Python\_basics.py}.

\end{enumerate}

\sphinxAtStartPar
We are going to write a simple program that converts an angle expressed in degrees to radians. In your \sphinxcode{\sphinxupquote{1\_Python\_basics.py}} file enter the following code (you can leave the text at the top of the file).

\begin{sphinxVerbatim}[commandchars=\\\{\}]
\PYG{n}{pi} \PYG{o}{=} \PYG{l+m+mf}{3.1415927}\PYG{p}{;}
\PYG{n}{angle\PYGZus{}in\PYGZus{}degrees} \PYG{o}{=} \PYG{l+m+mi}{45}\PYG{p}{;}
\PYG{n}{angle\PYGZus{}in\PYGZus{}radians} \PYG{o}{=} \PYG{n}{angle\PYGZus{}in\PYGZus{}degrees} \PYG{o}{*} \PYG{n}{pi} \PYG{o}{/} \PYG{l+m+mi}{180}\PYG{p}{;}
\end{sphinxVerbatim}

\sphinxAtStartPar
The first three lines should be familiar to you as we have declared the two variables \sphinxcode{\sphinxupquote{pi}} and \sphinxcode{\sphinxupquote{angle\_in\_degrees}} which are used to calculate the value of \sphinxcode{\sphinxupquote{angle\_in\_radians}}. Run the program by clicking on the play button or by pressing the F5 key. Not a lot happens apart from the text \sphinxcode{\sphinxupquote{runfile(...)}} appearing in the console. We need to output the value of the angle in radians so add the following command to your program.

\begin{sphinxVerbatim}[commandchars=\\\{\}]
\PYG{n+nb}{print}\PYG{p}{(}\PYG{n}{angle\PYGZus{}in\PYGZus{}radians}\PYG{p}{)}
\end{sphinxVerbatim}

\sphinxAtStartPar
Running your program now outputs the following to the console.

\begin{sphinxVerbatim}[commandchars=\\\{\}]
\PYG{l+m+mf}{0.785398175}
\end{sphinxVerbatim}


\bigskip\hrule\bigskip



\part{Exercise}
\label{\detokenize{_pages/1.5_Python_programs:exercise}}
\sphinxAtStartPar
Create a new Python file called \sphinxcode{\sphinxupquote{1\_Python\_basics\_exercises.py}} and save it to your OneDrive folder. Use it to answer the remaining exercises in this chapter.
\phantomsection \label{exercise:python-basic-programs-ex}

\begin{sphinxadmonition}{note}{Exercise 1.6.1}



\sphinxAtStartPar
Write a program similar to \sphinxcode{\sphinxupquote{python\sphinxhyphen{}variables\sphinxhyphen{}ex}} except that it converts a temperature from degrees Fahrenheit to degrees Centigrade and outputs the result to the console.

\sphinxAtStartPar
Use your program to convert the following to degrees Centigrade:
\begin{enumerate}
\sphinxsetlistlabels{\arabic}{enumi}{enumii}{}{.}%
\item {} 
\sphinxAtStartPar
100 \(^\circ\)F

\item {} 
\sphinxAtStartPar
0 \(^\circ\)F

\end{enumerate}
\end{sphinxadmonition}







\renewcommand{\indexname}{Index}
\printindex
\end{document}