%% Generated by Sphinx.
\def\sphinxdocclass{jupyterBook}
\documentclass[letterpaper,10pt,english]{jupyterBook}
\ifdefined\pdfpxdimen
   \let\sphinxpxdimen\pdfpxdimen\else\newdimen\sphinxpxdimen
\fi \sphinxpxdimen=.75bp\relax
\ifdefined\pdfimageresolution
    \pdfimageresolution= \numexpr \dimexpr1in\relax/\sphinxpxdimen\relax
\fi
%% let collapsible pdf bookmarks panel have high depth per default
\PassOptionsToPackage{bookmarksdepth=5}{hyperref}
%% turn off hyperref patch of \index as sphinx.xdy xindy module takes care of
%% suitable \hyperpage mark-up, working around hyperref-xindy incompatibility
\PassOptionsToPackage{hyperindex=false}{hyperref}
%% memoir class requires extra handling
\makeatletter\@ifclassloaded{memoir}
{\ifdefined\memhyperindexfalse\memhyperindexfalse\fi}{}\makeatother

\PassOptionsToPackage{warn}{textcomp}

\catcode`^^^^00a0\active\protected\def^^^^00a0{\leavevmode\nobreak\ }
\usepackage{cmap}
\usepackage{fontspec}
\defaultfontfeatures[\rmfamily,\sffamily,\ttfamily]{}
\usepackage{amsmath,amssymb,amstext}
\usepackage{polyglossia}
\setmainlanguage{english}



\setmainfont{FreeSerif}[
  Extension      = .otf,
  UprightFont    = *,
  ItalicFont     = *Italic,
  BoldFont       = *Bold,
  BoldItalicFont = *BoldItalic
]
\setsansfont{FreeSans}[
  Extension      = .otf,
  UprightFont    = *,
  ItalicFont     = *Oblique,
  BoldFont       = *Bold,
  BoldItalicFont = *BoldOblique,
]
\setmonofont{FreeMono}[
  Extension      = .otf,
  UprightFont    = *,
  ItalicFont     = *Oblique,
  BoldFont       = *Bold,
  BoldItalicFont = *BoldOblique,
]



\usepackage[Bjarne]{fncychap}
\usepackage[,numfigreset=1,mathnumfig]{sphinx}

\fvset{fontsize=\small}
\usepackage{geometry}


% Include hyperref last.
\usepackage{hyperref}
% Fix anchor placement for figures with captions.
\usepackage{hypcap}% it must be loaded after hyperref.
% Set up styles of URL: it should be placed after hyperref.
\urlstyle{same}


\usepackage{sphinxmessages}



        % Start of preamble defined in sphinx-jupyterbook-latex %
         \usepackage[Latin,Greek]{ucharclasses}
        \usepackage{unicode-math}
        % fixing title of the toc
        \addto\captionsenglish{\renewcommand{\contentsname}{Contents}}
        \hypersetup{
            pdfencoding=auto,
            psdextra
        }
        % End of preamble defined in sphinx-jupyterbook-latex %
        

\title{Python Variables}
\date{Sep 25, 2024}
\release{}
\author{Dr Jon Shiach}
\newcommand{\sphinxlogo}{\vbox{}}
\renewcommand{\releasename}{}
\makeindex
\begin{document}

\pagestyle{empty}
\sphinxmaketitle
\pagestyle{plain}
\sphinxtableofcontents
\pagestyle{normal}
\phantomsection\label{\detokenize{_pages/1.2_Python_variables::doc}}


\sphinxAtStartPar
A \sphinxstylestrong{variable} is a portion of memory uses to store a value. The Python syntax for declaring a variable is

\begin{sphinxVerbatim}[commandchars=\\\{\}]
\PYG{n}{variable\PYGZus{}name} \PYG{o}{=} \PYG{n}{value}
\end{sphinxVerbatim}

\sphinxAtStartPar
where \sphinxcode{\sphinxupquote{variable\_name}} is the name which is used to refer to the variable and \sphinxcode{\sphinxupquote{value}} is the value that is stored in the memory. Lets declare a variable and assign a value to it. In the console enter the following and press the enter key.

\begin{sphinxVerbatim}[commandchars=\\\{\}]
\PYG{n}{In} \PYG{p}{[}\PYG{l+m+mi}{17}\PYG{p}{]}\PYG{p}{:} \PYG{n}{a} \PYG{o}{=} \PYG{l+m+mi}{2}
\end{sphinxVerbatim}

\sphinxAtStartPar
You may be a bit disappointed to see that nothing happens. In fact Python has stored the value of 2 into the memory which can be accessed using the variable name \sphinxcode{\sphinxupquote{a}}. We can retrieve the value of our variable by typing \sphinxcode{\sphinxupquote{a}} into the console and press the enter key.

\begin{sphinxVerbatim}[commandchars=\\\{\}]
\PYG{n}{In} \PYG{p}{[}\PYG{l+m+mi}{18}\PYG{p}{]}\PYG{p}{:} \PYG{n}{a}
\PYG{n}{Out}\PYG{p}{[}\PYG{l+m+mi}{18}\PYG{p}{]}\PYG{p}{:} \PYG{l+m+mi}{2}
\end{sphinxVerbatim}

\sphinxAtStartPar
So Python has retrieved the value of \sphinxcode{\sphinxupquote{a}} for us. We can change the value of \sphinxcode{\sphinxupquote{a}} simply by assigning another value. In the console enter the following and press the enter key.

\begin{sphinxVerbatim}[commandchars=\\\{\}]
\PYG{n}{In} \PYG{p}{[}\PYG{l+m+mi}{19}\PYG{p}{]}\PYG{p}{:} \PYG{n}{a} \PYG{o}{=} \PYG{l+m+mi}{3}
\end{sphinxVerbatim}

\sphinxAtStartPar
and retrieve the value of \sphinxcode{\sphinxupquote{a}} again

\begin{sphinxVerbatim}[commandchars=\\\{\}]
\PYG{n}{In} \PYG{p}{[}\PYG{l+m+mi}{20}\PYG{p}{]}\PYG{p}{:} \PYG{n}{a}
\PYG{n}{Out}\PYG{p}{[}\PYG{l+m+mi}{20}\PYG{p}{]}\PYG{p}{:} \PYG{l+m+mi}{3}
\end{sphinxVerbatim}

\sphinxAtStartPar
Lets use variables to calculate the area of a rectangle. Enter the following into the console.

\begin{sphinxVerbatim}[commandchars=\\\{\}]
\PYG{n}{In} \PYG{p}{[}\PYG{l+m+mi}{21}\PYG{p}{]}\PYG{p}{:} \PYG{n}{width} \PYG{o}{=} \PYG{l+m+mi}{4}

\PYG{n}{In} \PYG{p}{[}\PYG{l+m+mi}{22}\PYG{p}{]}\PYG{p}{:} \PYG{n}{height} \PYG{o}{=} \PYG{l+m+mi}{3}

\PYG{n}{In} \PYG{p}{[}\PYG{l+m+mi}{23}\PYG{p}{]}\PYG{p}{:} \PYG{n}{area\PYGZus{}of\PYGZus{}rectangle} \PYG{o}{=} \PYG{n}{width} \PYG{o}{*} \PYG{n}{height}

\PYG{n}{In} \PYG{p}{[}\PYG{l+m+mi}{24}\PYG{p}{]}\PYG{p}{:} \PYG{n}{area\PYGZus{}of\PYGZus{}rectangle}
\PYG{n}{Out}\PYG{p}{[}\PYG{l+m+mi}{24}\PYG{p}{]}\PYG{p}{:} \PYG{l+m+mi}{12}
\end{sphinxVerbatim}

\sphinxAtStartPar
Here we have created two variables \sphinxcode{\sphinxupquote{width}} and \sphinxcode{\sphinxupquote{height}} to store the width and height of a rectangle. We then create a third variable \sphinxcode{\sphinxupquote{area\_of\_rectangle}} which stores the area of the rectangle calculated using the values stored in the other two variables. We then retrieve the area of the rectangle.

\sphinxAtStartPar
Python also allows us to declare multiple variables on a single line. Enter the following into the console.

\begin{sphinxVerbatim}[commandchars=\\\{\}]
\PYG{n}{In} \PYG{p}{[}\PYG{l+m+mi}{25}\PYG{p}{]}\PYG{p}{:} \PYG{n}{base}\PYG{p}{,} \PYG{n}{height} \PYG{o}{=} \PYG{l+m+mi}{2}\PYG{p}{,} \PYG{l+m+mi}{1}

\PYG{n}{In} \PYG{p}{[}\PYG{l+m+mi}{26}\PYG{p}{]}\PYG{p}{:} \PYG{n}{area\PYGZus{}of\PYGZus{}triangle} \PYG{o}{=} \PYG{n}{base} \PYG{o}{*} \PYG{n}{height} \PYG{o}{/} \PYG{l+m+mi}{2}

\PYG{n}{In} \PYG{p}{[}\PYG{l+m+mi}{27}\PYG{p}{]}\PYG{p}{:} \PYG{n}{area\PYGZus{}of\PYGZus{}triangle}
\PYG{n}{Out}\PYG{p}{[}\PYG{l+m+mi}{27}\PYG{p}{]}\PYG{p}{:} \PYG{l+m+mf}{1.0}
\end{sphinxVerbatim}

\sphinxAtStartPar
Here we have declared two variables \sphinxcode{\sphinxupquote{base}} and \sphinxcode{\sphinxupquote{height}} and set their values to 2 and 1 respectively (this overwrites the previous value of \sphinxcode{\sphinxupquote{height}}). Then we have calculated the area of a triangle and stored it in the variable \sphinxcode{\sphinxupquote{area\_of\_triangle}} and retrieved it.


\part{Variable names}
\label{\detokenize{_pages/1.2_Python_variables:variable-names}}
\sphinxAtStartPar
The choice of variable name is up to us but it must satisfy the following rules:
\begin{itemize}
\item {} 
\sphinxAtStartPar
a variable name must start with a letter or the underscore (\sphinxcode{\sphinxupquote{\_}}) character

\item {} 
\sphinxAtStartPar
a variable name can only contain alpha\sphinxhyphen{}numeric characters and underscores (\sphinxcode{\sphinxupquote{a}} \sphinxhyphen{} \sphinxcode{\sphinxupquote{z}}, \sphinxcode{\sphinxupquote{A}} \sphinxhyphen{} \sphinxcode{\sphinxupquote{Z}} and \sphinxcode{\sphinxupquote{\_}})

\item {} 
\sphinxAtStartPar
a variable name is case sensitive, i.e., \sphinxcode{\sphinxupquote{age}}, \sphinxcode{\sphinxupquote{Age}} and \sphinxcode{\sphinxupquote{AGE}} are three different variables

\item {} 
\sphinxAtStartPar
a variable name cannot by any of the Python keywords

\end{itemize}

\begin{sphinxadmonition}{important}{Important:}
\sphinxAtStartPar
Care must be taken when choosing variable names not to use a name already used by Python or a Python library. For example, if we use a variable name \sphinxcode{\sphinxupquote{print}} then we will not by able to use the \sphinxcode{\sphinxupquote{print()}} command (see below) since this will have been overwritten by our variable.
\end{sphinxadmonition}

\sphinxAtStartPar
An example of a variable name that would violate Python’s naming rules is \sphinxcode{\sphinxupquote{base length}}. Enter the following into the console

\begin{sphinxVerbatim}[commandchars=\\\{\}]
\PYG{n}{base} \PYG{n}{length} \PYG{o}{=} \PYG{l+m+mi}{2}
\end{sphinxVerbatim}

\sphinxAtStartPar
When you press enter the following is printed to the console

\begin{sphinxVerbatim}[commandchars=\\\{\}]
\PYG{n}{In} \PYG{p}{[}\PYG{l+m+mi}{28}\PYG{p}{]}\PYG{p}{:} \PYG{n}{base} \PYG{n}{length} \PYG{o}{=} \PYG{l+m+mi}{2}
  \PYG{n}{Cell} \PYG{n}{In}\PYG{p}{[}\PYG{l+m+mi}{28}\PYG{p}{]}\PYG{p}{,} \PYG{n}{line} \PYG{l+m+mi}{1}
    \PYG{n}{base} \PYG{n}{length} \PYG{o}{=} \PYG{l+m+mi}{2}
         \PYG{o}{\PYGZca{}}
\PYG{n+ne}{SyntaxError}\PYG{p}{:} \PYG{n}{invalid} \PYG{n}{syntax}
\end{sphinxVerbatim}

\sphinxAtStartPar
Here Python is telling us that our variable name is invalid syntax. We can correct this using the variable name \sphinxcode{\sphinxupquote{base\_length}}, enter the following into the console

\begin{sphinxVerbatim}[commandchars=\\\{\}]
\PYG{n}{In} \PYG{p}{[}\PYG{l+m+mi}{29}\PYG{p}{]}\PYG{p}{:} \PYG{n}{base\PYGZus{}length} \PYG{o}{=} \PYG{l+m+mi}{2}
\end{sphinxVerbatim}

\sphinxAtStartPar
which Python has happily dealt with. It is good practice to use variable names that are descriptive of what the variable represents, e.g., using \sphinxcode{\sphinxupquote{length}} for the length is easier to read and understand then \sphinxcode{\sphinxupquote{l}} (try to avoid lowercase \sphinxcode{\sphinxupquote{l}} are its easy to confuse then with the characters \sphinxcode{\sphinxupquote{1}} and \sphinxcode{\sphinxupquote{I}}). The use of underscores \sphinxcode{\sphinxupquote{\_}} in place of spaces in variable names is known as \sphinxstylestrong{pothole case} or \sphinxstylestrong{snake case} is common in Python as it makes code easier to read.


\bigskip\hrule\bigskip



\part{Exercise}
\label{\detokenize{_pages/1.2_Python_variables:exercise}}\phantomsection \label{exercise:python-variables-ex}

\begin{sphinxadmonition}{note}{Exercise 1.3.1}



\sphinxAtStartPar
The following formula converts a temperature from degree Fahrenheit to degrees Centigrade
\begin{equation*}
\begin{split} F = \frac{9}{5}C + 32.\end{split}
\end{equation*}
\sphinxAtStartPar
Using appropriate variable names, convert the following temperatures from degrees Fahrenheit to degrees centigrade:
\begin{enumerate}
\sphinxsetlistlabels{\arabic}{enumi}{enumii}{}{.}%
\item {} 
\sphinxAtStartPar
37.8\(^\circ\)C

\item {} 
\sphinxAtStartPar
100\(^\circ\)C

\item {} 
\sphinxAtStartPar
\(-\)40\(^\circ\)C

\end{enumerate}
\end{sphinxadmonition}







\renewcommand{\indexname}{Index}
\printindex
\end{document}