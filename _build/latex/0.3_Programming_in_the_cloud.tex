%% Generated by Sphinx.
\def\sphinxdocclass{jupyterBook}
\documentclass[letterpaper,10pt,english]{jupyterBook}
\ifdefined\pdfpxdimen
   \let\sphinxpxdimen\pdfpxdimen\else\newdimen\sphinxpxdimen
\fi \sphinxpxdimen=.75bp\relax
\ifdefined\pdfimageresolution
    \pdfimageresolution= \numexpr \dimexpr1in\relax/\sphinxpxdimen\relax
\fi
%% let collapsible pdf bookmarks panel have high depth per default
\PassOptionsToPackage{bookmarksdepth=5}{hyperref}
%% turn off hyperref patch of \index as sphinx.xdy xindy module takes care of
%% suitable \hyperpage mark-up, working around hyperref-xindy incompatibility
\PassOptionsToPackage{hyperindex=false}{hyperref}
%% memoir class requires extra handling
\makeatletter\@ifclassloaded{memoir}
{\ifdefined\memhyperindexfalse\memhyperindexfalse\fi}{}\makeatother

\PassOptionsToPackage{warn}{textcomp}

\catcode`^^^^00a0\active\protected\def^^^^00a0{\leavevmode\nobreak\ }
\usepackage{cmap}
\usepackage{fontspec}
\defaultfontfeatures[\rmfamily,\sffamily,\ttfamily]{}
\usepackage{amsmath,amssymb,amstext}
\usepackage{polyglossia}
\setmainlanguage{english}



\setmainfont{FreeSerif}[
  Extension      = .otf,
  UprightFont    = *,
  ItalicFont     = *Italic,
  BoldFont       = *Bold,
  BoldItalicFont = *BoldItalic
]
\setsansfont{FreeSans}[
  Extension      = .otf,
  UprightFont    = *,
  ItalicFont     = *Oblique,
  BoldFont       = *Bold,
  BoldItalicFont = *BoldOblique,
]
\setmonofont{FreeMono}[
  Extension      = .otf,
  UprightFont    = *,
  ItalicFont     = *Oblique,
  BoldFont       = *Bold,
  BoldItalicFont = *BoldOblique,
]



\usepackage[Bjarne]{fncychap}
\usepackage[,numfigreset=1,mathnumfig]{sphinx}

\fvset{fontsize=\small}
\usepackage{geometry}


% Include hyperref last.
\usepackage{hyperref}
% Fix anchor placement for figures with captions.
\usepackage{hypcap}% it must be loaded after hyperref.
% Set up styles of URL: it should be placed after hyperref.
\urlstyle{same}


\usepackage{sphinxmessages}



        % Start of preamble defined in sphinx-jupyterbook-latex %
         \usepackage[Latin,Greek]{ucharclasses}
        \usepackage{unicode-math}
        % fixing title of the toc
        \addto\captionsenglish{\renewcommand{\contentsname}{Contents}}
        \hypersetup{
            pdfencoding=auto,
            psdextra
        }
        % End of preamble defined in sphinx-jupyterbook-latex %
        

\title{Programming in the Cloud}
\date{Sep 25, 2024}
\release{}
\author{Dr Jon Shiach}
\newcommand{\sphinxlogo}{\vbox{}}
\renewcommand{\releasename}{}
\makeindex
\begin{document}

\pagestyle{empty}
\sphinxmaketitle
\pagestyle{plain}
\sphinxtableofcontents
\pagestyle{normal}
\phantomsection\label{\detokenize{_pages/0.3_Programming_in_the_cloud::doc}}


\sphinxAtStartPar
Programming in the cloud through a web browser is particularly useful for when you need to access Python or MATLAB from different locations or devices, including those that might not support local installation, such as tablets or shared computers.


\part{Replit}
\label{\detokenize{_pages/0.3_Programming_in_the_cloud:replit}}
\sphinxAtStartPar
Replit is an online integrated development environment (IDE) that supports various programming languages, including Python, that runs in the cloud.
\begin{enumerate}
\sphinxsetlistlabels{\arabic}{enumi}{enumii}{}{.}%
\item {} 
\sphinxAtStartPar
Go to https://replit.com and click on \sphinxstylestrong{Sign up for free}

\item {} 
\sphinxAtStartPar
Click on \sphinxstylestrong{Continue with email} and use your personal email account to create an account

\item {} 
\sphinxAtStartPar
Log into your account and click on \sphinxstylestrong{Create Repl} in the top\sphinxhyphen{}left hand corner of the page

\item {} 
\sphinxAtStartPar
Under \sphinxstylestrong{Languages \& Frameworks} click on \sphinxstylestrong{Python}

\item {} 
\sphinxAtStartPar
Enter a title (e.g., “Programming Skills”) and click on \sphinxstylestrong{Create Repl}

\end{enumerate}

\begin{figure}[htbp]
\centering

\noindent\sphinxincludegraphics[width=600\sphinxpxdimen]{{0_Replit}.png}
\end{figure}


\part{MATLAB Online}
\label{\detokenize{_pages/0.3_Programming_in_the_cloud:matlab-online}}
\sphinxAtStartPar
MATLAB Online is a web\sphinxhyphen{}based version of MATLAB, which allows you to run MATLAB code and use MATLAB features directly from a web browser without the need for local installation.

\sphinxAtStartPar
To access MATLAB online go to https://matlab.mathworks.com/ and click on \sphinxstylestrong{Open MATLAB Online}

\begin{figure}[htbp]
\centering

\noindent\sphinxincludegraphics[width=600\sphinxpxdimen]{{0_MATLAB_online}.png}
\end{figure}


\part{Google Colab}
\label{\detokenize{_pages/0.3_Programming_in_the_cloud:google-colab}}
\sphinxAtStartPar
Google Colab is a free, cloud\sphinxhyphen{}based platform provided by Google that allows users to write and execute Python code in an interactive notebook environment.

\begin{figure}[htbp]
\centering

\noindent\sphinxincludegraphics[width=600\sphinxpxdimen]{{0_Google_colab}.png}
\end{figure}







\renewcommand{\indexname}{Index}
\printindex
\end{document}