%% Generated by Sphinx.
\def\sphinxdocclass{jupyterBook}
\documentclass[letterpaper,10pt,english]{jupyterBook}
\ifdefined\pdfpxdimen
   \let\sphinxpxdimen\pdfpxdimen\else\newdimen\sphinxpxdimen
\fi \sphinxpxdimen=.75bp\relax
\ifdefined\pdfimageresolution
    \pdfimageresolution= \numexpr \dimexpr1in\relax/\sphinxpxdimen\relax
\fi
%% let collapsible pdf bookmarks panel have high depth per default
\PassOptionsToPackage{bookmarksdepth=5}{hyperref}
%% turn off hyperref patch of \index as sphinx.xdy xindy module takes care of
%% suitable \hyperpage mark-up, working around hyperref-xindy incompatibility
\PassOptionsToPackage{hyperindex=false}{hyperref}
%% memoir class requires extra handling
\makeatletter\@ifclassloaded{memoir}
{\ifdefined\memhyperindexfalse\memhyperindexfalse\fi}{}\makeatother

\PassOptionsToPackage{warn}{textcomp}

\catcode`^^^^00a0\active\protected\def^^^^00a0{\leavevmode\nobreak\ }
\usepackage{cmap}
\usepackage{fontspec}
\defaultfontfeatures[\rmfamily,\sffamily,\ttfamily]{}
\usepackage{amsmath,amssymb,amstext}
\usepackage{polyglossia}
\setmainlanguage{english}



\setmainfont{FreeSerif}[
  Extension      = .otf,
  UprightFont    = *,
  ItalicFont     = *Italic,
  BoldFont       = *Bold,
  BoldItalicFont = *BoldItalic
]
\setsansfont{FreeSans}[
  Extension      = .otf,
  UprightFont    = *,
  ItalicFont     = *Oblique,
  BoldFont       = *Bold,
  BoldItalicFont = *BoldOblique,
]
\setmonofont{FreeMono}[
  Extension      = .otf,
  UprightFont    = *,
  ItalicFont     = *Oblique,
  BoldFont       = *Bold,
  BoldItalicFont = *BoldOblique,
]



\usepackage[Bjarne]{fncychap}
\usepackage[,numfigreset=1,mathnumfig]{sphinx}

\fvset{fontsize=\small}
\usepackage{geometry}


% Include hyperref last.
\usepackage{hyperref}
% Fix anchor placement for figures with captions.
\usepackage{hypcap}% it must be loaded after hyperref.
% Set up styles of URL: it should be placed after hyperref.
\urlstyle{same}


\usepackage{sphinxmessages}



        % Start of preamble defined in sphinx-jupyterbook-latex %
         \usepackage[Latin,Greek]{ucharclasses}
        \usepackage{unicode-math}
        % fixing title of the toc
        \addto\captionsenglish{\renewcommand{\contentsname}{Contents}}
        \hypersetup{
            pdfencoding=auto,
            psdextra
        }
        % End of preamble defined in sphinx-jupyterbook-latex %
        

\title{Printing output}
\date{Sep 25, 2024}
\release{}
\author{Dr Jon Shiach}
\newcommand{\sphinxlogo}{\vbox{}}
\renewcommand{\releasename}{}
\makeindex
\begin{document}

\pagestyle{empty}
\sphinxmaketitle
\pagestyle{plain}
\sphinxtableofcontents
\pagestyle{normal}
\phantomsection\label{\detokenize{_pages/1.6_Printing_output::doc}}


\sphinxAtStartPar
In previous sections have used the \sphinxcode{\sphinxupquote{print()}} function to print the value of a variable to the console. This can be used to print a variety of data types, for example the command

\begin{sphinxVerbatim}[commandchars=\\\{\}]
\PYG{n+nb}{print}\PYG{p}{(}\PYG{l+s+s2}{\PYGZdq{}}\PYG{l+s+s2}{hello world}\PYG{l+s+s2}{\PYGZdq{}}\PYG{p}{)}
\end{sphinxVerbatim}

\sphinxAtStartPar
will print the text string \sphinxcode{\sphinxupquote{hello world}}. Lets print a short header that tells someone what our program does. Edit your program so that the following appears above the \sphinxcode{\sphinxupquote{pi}} variable declaration

\begin{sphinxVerbatim}[commandchars=\\\{\}]
\PYG{n+nb}{print}\PYG{p}{(}\PYG{l+s+s2}{\PYGZdq{}}\PYG{l+s+s2}{Degrees to radians conversion}\PYG{l+s+s2}{\PYGZdq{}}\PYG{p}{)}
\PYG{n+nb}{print}\PYG{p}{(}\PYG{l+s+s2}{\PYGZdq{}}\PYG{l+s+s2}{\PYGZhy{}\PYGZhy{}\PYGZhy{}\PYGZhy{}\PYGZhy{}\PYGZhy{}\PYGZhy{}\PYGZhy{}\PYGZhy{}\PYGZhy{}\PYGZhy{}\PYGZhy{}\PYGZhy{}\PYGZhy{}\PYGZhy{}\PYGZhy{}\PYGZhy{}\PYGZhy{}\PYGZhy{}\PYGZhy{}\PYGZhy{}\PYGZhy{}\PYGZhy{}\PYGZhy{}\PYGZhy{}\PYGZhy{}\PYGZhy{}\PYGZhy{}\PYGZhy{}}\PYG{l+s+s2}{\PYGZdq{}}\PYG{p}{)}
\end{sphinxVerbatim}

\sphinxAtStartPar
Running your program you should see the following printed to the console.

\begin{sphinxVerbatim}[commandchars=\\\{\}]
Degrees to radians conversion
\PYGZhy{}\PYGZhy{}\PYGZhy{}\PYGZhy{}\PYGZhy{}\PYGZhy{}\PYGZhy{}\PYGZhy{}\PYGZhy{}\PYGZhy{}\PYGZhy{}\PYGZhy{}\PYGZhy{}\PYGZhy{}\PYGZhy{}\PYGZhy{}\PYGZhy{}\PYGZhy{}\PYGZhy{}\PYGZhy{}\PYGZhy{}\PYGZhy{}\PYGZhy{}\PYGZhy{}\PYGZhy{}\PYGZhy{}\PYGZhy{}\PYGZhy{}\PYGZhy{}
0.785398175
\end{sphinxVerbatim}

\sphinxAtStartPar
Note that each time the \sphinxcode{\sphinxupquote{print()}} function is called the text is printed on a new line.


\bigskip\hrule\bigskip



\part{Printing text and numbers}
\label{\detokenize{_pages/1.6_Printing_output:printing-text-and-numbers}}
\sphinxAtStartPar
So now our program has three print commands, two that prints some text and another that prints a floating point number. Wouldn’t it be nice if we could print text and numbers at the same time. This is known as \sphinxstylestrong{formatted output} and we can do this by putting an \sphinxcode{\sphinxupquote{f}} between the opening bracket and the double quotes \sphinxcode{\sphinxupquote{"}} and a variable in curly brackets \sphinxcode{\sphinxupquote{\{...\}}}, i.e.,

\begin{sphinxVerbatim}[commandchars=\\\{\}]
\PYG{n+nb}{print}\PYG{p}{(}\PYG{l+s+sa}{f}\PYG{l+s+s2}{\PYGZdq{}}\PYG{l+s+s2}{some text }\PYG{l+s+si}{\PYGZob{}}\PYG{n}{variable}\PYG{l+s+si}{\PYGZcb{}}\PYG{l+s+s2}{\PYGZdq{}}\PYG{p}{)}
\end{sphinxVerbatim}

\sphinxAtStartPar
Edit your program so that the following print commands replace the command used to print the angle in radians.

\begin{sphinxVerbatim}[commandchars=\\\{\}]
\PYG{n+nb}{print}\PYG{p}{(}\PYG{l+s+sa}{f}\PYG{l+s+s2}{\PYGZdq{}}\PYG{l+s+s2}{angle in degrees = }\PYG{l+s+si}{\PYGZob{}}\PYG{n}{angle\PYGZus{}in\PYGZus{}degrees}\PYG{l+s+si}{\PYGZcb{}}\PYG{l+s+s2}{\PYGZdq{}}\PYG{p}{)}
\PYG{n+nb}{print}\PYG{p}{(}\PYG{l+s+sa}{f}\PYG{l+s+s2}{\PYGZdq{}}\PYG{l+s+s2}{angle in radians = }\PYG{l+s+si}{\PYGZob{}}\PYG{n}{angle\PYGZus{}in\PYGZus{}radians}\PYG{l+s+si}{\PYGZcb{}}\PYG{l+s+s2}{\PYGZdq{}}\PYG{p}{)}
\end{sphinxVerbatim}

\sphinxAtStartPar
Running your program you should see the following printed to the console.

\begin{sphinxVerbatim}[commandchars=\\\{\}]
Degrees to radians conversion
angle in degrees = 45
angle in radians = 0.785398175
\end{sphinxVerbatim}


\bigskip\hrule\bigskip



\part{Format specifier}
\label{\detokenize{_pages/1.6_Printing_output:format-specifier}}
\sphinxAtStartPar
In our program the angle in radians is printed using 9 decimal places. We probably don’t need this level of accuracy so lets reduce the number of decimal places outputted to 3 by editing the print commands so that they look like the following.

\begin{sphinxVerbatim}[commandchars=\\\{\}]
\PYG{n+nb}{print}\PYG{p}{(}\PYG{l+s+sa}{f}\PYG{l+s+s2}{\PYGZdq{}}\PYG{l+s+s2}{angle in degrees = }\PYG{l+s+si}{\PYGZob{}}\PYG{n}{angle\PYGZus{}in\PYGZus{}degrees}\PYG{l+s+si}{:}\PYG{l+s+s2}{6.3f}\PYG{l+s+si}{\PYGZcb{}}\PYG{l+s+s2}{\PYGZdq{}}\PYG{p}{)}
\PYG{n+nb}{print}\PYG{p}{(}\PYG{l+s+sa}{f}\PYG{l+s+s2}{\PYGZdq{}}\PYG{l+s+s2}{angle in radians = }\PYG{l+s+si}{\PYGZob{}}\PYG{n}{angle\PYGZus{}in\PYGZus{}radians}\PYG{l+s+si}{:}\PYG{l+s+s2}{6.3f}\PYG{l+s+si}{\PYGZcb{}}\PYG{l+s+s2}{\PYGZdq{}}\PYG{p}{)}
\end{sphinxVerbatim}

\sphinxAtStartPar
Running your program you should see the following printed to the console.

\begin{sphinxVerbatim}[commandchars=\\\{\}]
Degrees to radians conversion
\PYGZhy{}\PYGZhy{}\PYGZhy{}\PYGZhy{}\PYGZhy{}\PYGZhy{}\PYGZhy{}\PYGZhy{}\PYGZhy{}\PYGZhy{}\PYGZhy{}\PYGZhy{}\PYGZhy{}\PYGZhy{}\PYGZhy{}\PYGZhy{}\PYGZhy{}\PYGZhy{}\PYGZhy{}\PYGZhy{}\PYGZhy{}\PYGZhy{}\PYGZhy{}\PYGZhy{}\PYGZhy{}\PYGZhy{}\PYGZhy{}\PYGZhy{}\PYGZhy{}
angle in degrees = 45.000
angle in radians =  0.785
\end{sphinxVerbatim}

\sphinxAtStartPar
Notice that the angles have now been printed using 3 decimal places. The \sphinxcode{\sphinxupquote{6.3f}} is an example of a format specifier which controls how the values of the variables are printed. Here we have told Python to print a float value that uses a total of 6 character spaces (including the decimal point) with 3 significant figures following the decimal point.

\sphinxAtStartPar
The different types of format specifiers are shown in \hyperref[\detokenize{_pages/1.6_Printing_output:format-specifiers-table}]{Table \ref{\detokenize{_pages/1.6_Printing_output:format-specifiers-table}}}.


\begin{savenotes}\sphinxattablestart
\centering
\sphinxcapstartof{table}
\sphinxthecaptionisattop
\sphinxcaption{Format specifiers}\label{\detokenize{_pages/1.6_Printing_output:format-specifiers-table}}
\sphinxaftertopcaption
\begin{tabulary}{\linewidth}[t]{|T|T|T|T|}
\hline
\sphinxstyletheadfamily 
\sphinxAtStartPar
Data type
&\sphinxstyletheadfamily 
\sphinxAtStartPar
Specifier
&\sphinxstyletheadfamily 
\sphinxAtStartPar
Python code
&\sphinxstyletheadfamily 
\sphinxAtStartPar
Output
\\
\hline
\sphinxAtStartPar
integer
&
\sphinxAtStartPar
\sphinxcode{\sphinxupquote{d}}
&
\sphinxAtStartPar
\sphinxcode{\sphinxupquote{print(f"\{2:3d\}")}}
&
\sphinxAtStartPar
\sphinxcode{\sphinxupquote{\#\#2}} (two blank spaces followed by \sphinxcode{\sphinxupquote{2}})
\\
\hline
\sphinxAtStartPar
float
&
\sphinxAtStartPar
\sphinxcode{\sphinxupquote{f}}
&
\sphinxAtStartPar
\sphinxcode{\sphinxupquote{print(f"\{1/3:10.3f\}")}}
&
\sphinxAtStartPar
\sphinxcode{\sphinxupquote{\#\#\#\#\#0.333}}
\\
\hline
\sphinxAtStartPar
Scientific notation
&
\sphinxAtStartPar
\sphinxcode{\sphinxupquote{e}}
&
\sphinxAtStartPar
\sphinxcode{\sphinxupquote{print(f"\{123456:10.2e\}")}}
&
\sphinxAtStartPar
\sphinxcode{\sphinxupquote{\#\#1.23e+05}} (equivalent to \(1.23\times 10^{5}\))
\\
\hline
\sphinxAtStartPar
String
&
\sphinxAtStartPar
\sphinxcode{\sphinxupquote{s}}
&
\sphinxAtStartPar
\sphinxcode{\sphinxupquote{print(f"\{'hello':10s\} world")}}
&
\sphinxAtStartPar
\sphinxcode{\sphinxupquote{hello\#\#\#\#\#\# world}}
\\
\hline
\end{tabulary}
\par
\sphinxattableend\end{savenotes}

\sphinxAtStartPar
If the number to the left of the decimal point is zero, Python will use the smallest number of spaces required to print the number, e.g., \sphinxcode{\sphinxupquote{print(f"\{1.23456:0.2f\}")}} will print \sphinxcode{\sphinxupquote{1.23}}.


\bigskip\hrule\bigskip



\part{The newline character}
\label{\detokenize{_pages/1.6_Printing_output:the-newline-character}}
\sphinxAtStartPar
The newline character, \sphinxcode{\sphinxupquote{\textbackslash{}n}}, is used to instruct Python to print the rest of the string on new line. For example

\begin{sphinxVerbatim}[commandchars=\\\{\}]
\PYG{n+nb}{print}\PYG{p}{(}\PYG{l+s+s2}{\PYGZdq{}}\PYG{l+s+s2}{This text }\PYG{l+s+se}{\PYGZbs{}n}\PYG{l+s+s2}{is printed}\PYG{l+s+se}{\PYGZbs{}n}\PYG{l+s+se}{\PYGZbs{}n}\PYG{l+s+s2}{on multiple lines }\PYG{l+s+se}{\PYGZbs{}n}\PYG{l+s+se}{\PYGZbs{}n}\PYG{l+s+se}{\PYGZbs{}n}\PYG{l+s+s2}{using a single print command.}\PYG{l+s+s2}{\PYGZdq{}}\PYG{p}{)}
\end{sphinxVerbatim}

\sphinxAtStartPar
will print

\begin{sphinxVerbatim}[commandchars=\\\{\}]
This text 
is printed

on multiple lines 


using a single print command.
\end{sphinxVerbatim}

\sphinxAtStartPar
Lets print the header using a single \sphinxcode{\sphinxupquote{print()}} command. Edit your program so that the first two \sphinxcode{\sphinxupquote{print()}} commands are replaced by the following.

\begin{sphinxVerbatim}[commandchars=\\\{\}]
\PYG{n+nb}{print}\PYG{p}{(}\PYG{l+s+s2}{\PYGZdq{}}\PYG{l+s+s2}{Degrees to radians conversion}\PYG{l+s+se}{\PYGZbs{}n}\PYG{l+s+s2}{\PYGZhy{}\PYGZhy{}\PYGZhy{}\PYGZhy{}\PYGZhy{}\PYGZhy{}\PYGZhy{}\PYGZhy{}\PYGZhy{}\PYGZhy{}\PYGZhy{}\PYGZhy{}\PYGZhy{}\PYGZhy{}\PYGZhy{}\PYGZhy{}\PYGZhy{}\PYGZhy{}\PYGZhy{}\PYGZhy{}\PYGZhy{}\PYGZhy{}\PYGZhy{}\PYGZhy{}\PYGZhy{}\PYGZhy{}\PYGZhy{}\PYGZhy{}\PYGZhy{}}\PYG{l+s+s2}{\PYGZdq{}}\PYG{p}{)}
\end{sphinxVerbatim}

\sphinxAtStartPar
Running your program you should see that the console output has not changed.


\bigskip\hrule\bigskip



\part{Escape characters}
\label{\detokenize{_pages/1.6_Printing_output:escape-characters}}
\sphinxAtStartPar
Some characters cannot be contained within a string, e.g., \sphinxcode{\sphinxupquote{"}}. To print such a character we can use an escape character which is a backslash \sphinxcode{\sphinxupquote{\textbackslash{}}} followed by the character we want to print. For example, to print

\begin{sphinxVerbatim}[commandchars=\\\{\}]
The feature film \PYGZdq{}Monty Python\PYGZsq{}s Life of Brian\PYGZdq{} was released in 1979.
\end{sphinxVerbatim}

\sphinxAtStartPar
we could use

\begin{sphinxVerbatim}[commandchars=\\\{\}]
\PYG{n+nb}{print}\PYG{p}{(}\PYG{l+s+sa}{f}\PYG{l+s+s2}{\PYGZdq{}}\PYG{l+s+s2}{The feature film }\PYG{l+s+se}{\PYGZbs{}\PYGZdq{}}\PYG{l+s+s2}{Monty Python}\PYG{l+s+s2}{\PYGZsq{}}\PYG{l+s+s2}{s Life of Brian}\PYG{l+s+se}{\PYGZbs{}\PYGZdq{}}\PYG{l+s+s2}{ was released in 1979.}\PYG{l+s+s2}{\PYGZdq{}}\PYG{p}{)}
\end{sphinxVerbatim}


\bigskip\hrule\bigskip



\part{Exercise}
\label{\detokenize{_pages/1.6_Printing_output:exercise}}\phantomsection \label{exercise:python-printing-ex}

\begin{sphinxadmonition}{note}{Exercise 1.7.1}



\sphinxAtStartPar
Write a program that determines the number of years, weeks, days, minutes and seconds that are in a given number of seconds. Your program should print out the result as a single sentence that mixes text and numbers. For example, if the given number of seconds is 1 million, then your program should print

\begin{sphinxVerbatim}[commandchars=\\\{\}]
There are 0 years, 1 weeks, 4 days, 13 hours, 46 minutes and 40 seconds in 1000000 seconds.
\end{sphinxVerbatim}

\sphinxAtStartPar
Use your program to print the number of years, weeks, days, hours and minutes in 1 billion seconds.
\end{sphinxadmonition}







\renewcommand{\indexname}{Index}
\printindex
\end{document}